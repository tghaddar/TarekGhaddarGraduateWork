%% This is file `jcomp-template.tex',
%% 
%% Copyright 2017 Elsevier Ltd
%% 
%% This file is part of the 'Elsarticle Bundle'.
%% ---------------------------------------------
%% 
%% It may be distributed under the conditions of the LaTeX Project Public
%% License, either version 1.2 of this license or (at your option) any
%% later version.  The latest version of this license is in
%%    http://www.latex-project.org/lppl.txt
%% and version 1.2 or later is part of all distributions of LaTeX
%% version 1999/12/01 or later.
%% 
%% The list of all files belonging to the 'Elsarticle Bundle' is
%% given in the file `manifest.txt'.
%% 
%% Template article for Elsevier's document class `elsarticle'
%% with harvard style bibliographic references
%%
%% $Id: jcomp-template.tex 100 2017-07-14 13:15:12Z rishi $
%%
%% Use the option review to obtain double line spacing
%\documentclass[times,review,preprint,authoryear]{elsarticle}

%% Use the options `twocolumn,final' to obtain the final layout
%% Use longtitle option to break abstract to multiple pages if overfull.
%% For Review pdf (With double line spacing)
%\documentclass[times,twocolumn,review]{elsarticle}
%% For abstracts longer than one page.
%\documentclass[times,twocolumn,review,longtitle]{elsarticle}
%% For Review pdf without preprint line
%\documentclass[times,twocolumn,review,nopreprintline]{elsarticle}
%% Final pdf
\documentclass[times,final]{elsarticle}
%%
%\documentclass[times,twocolumn,final,longtitle]{elsarticle}
%%


%% Stylefile to load JCOMP template
\usepackage{jcomp}
\usepackage{framed,multirow}

%% The amssymb package provides various useful mathematical symbols
\usepackage{amssymb}
\usepackage{latexsym}

% Following three lines are needed for this document.
% If you are not loading colors or url, then these are
% not required.
\usepackage{url}
\usepackage{xcolor}
\definecolor{newcolor}{rgb}{.8,.349,.1}

\journal{Journal of Computational Physics}

\begin{document}

\verso{Given-name Surname \textit{etal}}

\begin{frontmatter}

\title{Choosing optimal partitions for massively parallel transport sweeps on unstructured grids}%\tnoteref{tnote1}}%
%\tnotetext[tnote1]{This is an example for title footnote coding.}

\author[1]{Tarek Habib \snm{Ghaddar}}%\corref{cor1}}
%\cortext[cor1]{Tarek Gahddar: 
%  Tel.: +0-000-000-0000;  
%  fax: +0-000-000-0000;}
\author[1]{Given-name2 \snm{Ragusa}}
%\fntext[fn1]{This is author footnote for second author.}  

\address[1]{Department of Nuclear Engineering, Texas A\&M University, College Station, TX 77843, USA}

\received{1 May 2013}
\finalform{10 May 2013}
\accepted{13 May 2013}
\availableonline{15 May 2013}
\communicated{S. Sarkar}


\begin{abstract}
%%%
A concise and factual abstract is required. The abstract should state briefly the purpose of the research, the principal results and major conclusions. An abstract is often presented separately from the article, so it must be able to stand alone. For this reason, References should be avoided, but if essential, then cite the author(s) and year(s). Also, non-standard or uncommon abbreviations should be avoided, but if essential they must be defined at their first mention in the abstract itself.
%%%%
\end{abstract}

%\begin{keyword}
%% MSC codes here, in the form: \MSC code \sep code
%% or \MSC[2008] code \sep code (2000 is the default)
%\MSC 41A05\sep 41A10\sep 65D05\sep 65D17
%% Keywords
%\KWD Keyword1\sep Keyword2\sep Keyword3
%\end{keyword}

\end{frontmatter}

%\linenumbers

%% main text
\section{Introduction}

The field of radiation transport studies the distribution of radiation (particle/energy) throughout a six-dimensional phase-space consisting of time, space, energy, and direction. 
Radiation transport is commonly used in reactor physics, medical physics, radiation shielding, criticality safety applications, stellar atmospheres, as well as the general simulation of radiation in different environments. 
Radiation transport is described by the Boltzmann transport equation \cite{bell_glasstone,zweifel,davison,duderstadt} and can be solved stochastically or deterministically.

The work presented in this paper deals with the transport equation discretized using multigroup in energy, discrete-ordinates (collocation) in angle, and discontinuous finite elements in space. 
The resulting discrete system of equations is amenable to iterative techniques whereby the streaming+total interaction operator need not to be assembled, generating large memory savings while being computationally very efficient.  
Essentially, this allows for a matrix-free approach, termed ``transport sweeps'' in the radiation transport community, where small local system of equations are solved cell-by-cell, for each angular direction and energy group, in a fashion that follows the flow of radiation in the computational mesh.

\section{Unstructured meshing and load balancing in PDT}

\section{
\section*{Acknowledgments}
Acknowledgments should be inserted at the end of the paper, before the
references, not as a footnote to the title. Use the unnumbered
Acknowledgements Head style for the Acknowledgments heading.

\section*{References}

Please ensure that every reference cited in the text is also present in
the reference list (and vice versa).

\section*{\itshape Reference style}

Text: All citations in the text should refer to:
\begin{enumerate}
\item Single author: the author's name (without initials, unless there
is ambiguity) and the year of publication;
\item Two authors: both authors' names and the year of publication;
\item Three or more authors: first author's name followed by `et al.'
and the year of publication.
\end{enumerate}
Citations may be made directly (or parenthetically). Groups of
references should be listed first alphabetically, then chronologically.

%%Vancouver style references.
\bibliographystyle{model1-num-names}
\bibliography{refs}

\end{document}

%%
