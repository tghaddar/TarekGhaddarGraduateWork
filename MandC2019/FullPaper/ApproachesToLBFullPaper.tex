%%%%%%%%%%%%%%%%%%%%%%%%%%%%%%%%%%%%%%%%%%%%%%%%%%%%%%%%%%%%%%%%%%%%%
%
%  This is a sample LaTeX input file for your contribution to 
%  the M&C2019 topical meeting.
%
%  Please use it as a template for your full paper 
%    Accompanying/related file(s) include: 
%       1. Document class/format file: mandc.cls
%       2. Sample Postscript Figure:   figure.pdf
%       3. A PDF file showing the desired appearance: mandc2019_template.pdf
%       4. cites.sty and citesort.sty that might be needed by some users 
%    Direct questions about these files to: palmert@engr.orst.edu
%											mark.dehart@inl.gov
%
%    Notes: 
%      (1) You can use the "dvips" utility to convert .dvi 
%          files to PostScript.  Then, use either Acrobat 
%          Distiller or "ps2pdf" to convert to PDF format. 
%      (2) Different versions of LaTeX have been observed to 
%          shift the page down, causing improper margins.
%          If this occurs, adjust the "topmargin" value in the
%          physor2018.cls file to achieve the proper margins. 
%
%%%%%%%%%%%%%%%%%%%%%%%%%%%%%%%%%%%%%%%%%%%%%%%%%%%%%%%%%%%%%%%%%%%%%


%%%%%%%%%%%%%%%%%%%%%%%%%%%%%%%%%%%%%%%%%%%%%%%%%%%%%%%%%%%%%%%%%%%%%
\documentclass[letterpaper]{mandc2019}
%
%  various packages that you may wish to activate for usage 
\usepackage{tabls}
\usepackage{cites}
\usepackage{epsf}
\usepackage{appendix}
\usepackage{ragged2e}
\usepackage[top=1in, bottom=1.in, left=1.in, right=1.in]{geometry}
\usepackage{enumitem}
\setlist[itemize]{leftmargin=*}
\usepackage{caption}
\captionsetup{width=1.0\textwidth,font={bf,normalsize},skip=0.3cm,within=none,justification=centering}

\newcommand{\tcr}[1]{\textcolor{red}{#1}}
\newcommand{\vr}{\vec{r}}
\newcommand{\vo}{\vec{\Omega}}

%\usepackage[justification=centering]{caption}

%
% Define title...
%
\title{APPROACHES TO LOAD BALANCING \\
MASSIVELY PARALLEL TRANSPORT SWEEPS\\
ON UNSTRUCTURED GRIDS}
%
% ...and authors
%
\author{%
  % FIRST AUTHORS 
  %
  \textbf{Tarek Habib Ghaddar$^1$, Jean C. Ragusa$^1$, Jan Vermaak$^1$, Marvin L. Adams$^1$} \\
$^1$Dept. of Nuclear Engineering, Texas  A\&M University \\
  College Station, TX, 77843-3133 \\ 
     \\
  \url{tghaddar@tamu.edu}, \url{jean.ragusa@tamu.edu}, \url{janv@tamu.edu}, \url{mladams@tamu.edu}
}
%
% Insert authors' names and short version of title in lines below
%
\newcommand{\authorHead}      % Author's names here use et al. if more than 3
           {First author}  
\newcommand{\shortTitle}      % Short title here (Shorten to fit all into a single line)
           {Paper Title }  
%%%%%%%%%%%%%%%%%%%%%%%%%%%%%%%%%%%%%%%%%%%%%%%%%%%%%%%%%%%%%%%%%%%%%
%
%   BEGIN DOCUMENT
%
%%%%%%%%%%%%%%%%%%%%%%%%%%%%%%%%%%%%%%%%%%%%%%%%%%%%%%%%%%%%%%%%%%%%%
\begin{document}
\maketitle
\justify 

\begin{abstract}
  Use 8.5$\times$11 paper size, with 1'' margins on all sides.  A required 200-250 
  word abstract starts on this line.  Leave two blank lines before ``ABSTRACT''
  and one after.  Use 11 point Times New Roman here and single 
  spacing. The abstract is a very brief summary highlighting main 
  accomplishments, what is new, and how it relates to the state-of-the-art.
\end{abstract}
\keywords{list of three to five key words}

\section{INTRODUCTION} 
The steady-state neutron transport equation describes the behavior of neutrons in a medium and is given by Eq. (\ref{continuous transport}):
\begin{multline}
\vo \cdot \vec \nabla \psi(\vr,E,\vo) + \Sigma_t(\vr,E) \psi(\vr,E,\vo)  = \\ 
\int_{0}^{\infty}dE' \int_{4\pi}d\Omega' \Sigma_s(\vr,E'\to E, \Omega'\to\Omega)\psi(\vr,E',\vo')  + S_{ext}(\vr,E,\vo) ,
\label{continuous transport}
\end{multline}
where $\vec{\Omega}\cdot \vec\nabla\psi$ is the leakage term and $\Sigma_t\psi$ is the total collision term (absorption, outscatter, and within group scattering). These represent the loss terms of the neutron transport equation. The right hand side of Eq.~\eqref{continuous transport} represents the gain terms, where $S_{ext}$ is the external source of neutrons and $\int_{0}^{\infty}dE'\int_{4\pi}d\Omega'\Sigma_s(E'\to E, \Omega'\to\Omega)\psi(\vr,E',\vo')$ is the inscatter term, which represents all neutrons scattering from energy $E'$ and direction $\vo'$ into $dE$ about energy $E$ and $d\Omega$ about direction $\vo$. 

Eq. (\ref{angle}) is obtained by assuming isotropic scattering, discretizing in energy, and discretizing in angle using the discrete ordinates method\cite{discrete ordinates}:
\begin{equation}
\vo_m \cdot \vec \nabla \psi_{g,m}(\vr) +\Sigma_{t,g}(\vr) \psi_{g,m}(\vr)  = \frac{1}{4\pi}\sum_{g^{\prime}}\Sigma_{s,g^{\prime}\to g}(\vr)\phi_{g^{\prime}}(\vr) + S_{ext,g,m}(\vr).
\label{angle}
\end{equation}

This allows us to solve a sequence of transport equations for one group and direction at a time. The discrete form of the transport equation is solved via source iteration, described in Eq. (\ref{source iteration}):
\begin{equation}
\vo_m \cdot \vec\nabla \psi_m^{(l+1)}(\vr) + \Sigma_t \psi_m^{(l+1)}(\vr) = q_m^{(l)}(\vr),
\label{source iteration}
\end{equation}
where the gain terms terms have been combined into one general source term, $q_m$. The angular flux of iteration $(l+1)$ is calculated using the $(l^{th})$ value of the scalar flux. 

After the angular and energy dependence have been accounted for, Eq. (\ref{source iteration}) must be discretized in space as well. This is done by meshing the domain and utilizing one of three popular discretization techniques: finite difference\cite{fd}, finite volume\cite{fd}, or discontinuous finite element\cite{Reed}, allowing one cell at a time to be solved. The solution across a cell interface is connected based on an upwind approach, where face outflow radiation becomes face inflow radiation for the downwind cells. 

PDT, Texas A\&M's massively parallel deterministic transport code, employs the transport sweep by with the process described above, using a finite element mesh and solving the discrete ordinates form of the transport equation using a discontinuous finite element approach and source iteration\cite{mpadams15, mpadams13}. Initially, PDT only swept on structured, logically Cartesian meshes. As the need to solve problems with more complex geometries arose, PDT added a support for arbitrary polyhedral unstructured meshes. However, this introduced imbalanced partitions, causing runtimes of problems to become unmanageable.


\section{SECOND OR SUBSEQUENT MAJOR HEADING} 
\label{sec:first}

This is Section~\ref{sec:first}. It is followed by a subsection, that is, 
\ref{sec:second}. The style for subsection titles and all text in this template is defined 
in the \emph{mandc2019.cls} file.  Make sure to avoid widow/orphan lines.

\subsection{Subsection Title: First Character of Each Non-trivial Word is Uppercase} 
\label{sec:second}

Double-space before and after secondary titles is automatic.  Figures and 
tables should appear as close as possible to where they are first
cited, e.g., Fig.~\ref{fig:amdahl}, in the text.  Figures are numbered in Arabic 
numerals, with the caption centered below the figure, in \textbf{boldface}. For a better 
arrangement it is strongly recommended that all the figures must be placed in the``Figures'' 
folder. Triple-space before the figure, and after the figure caption.

\begin{figure}[!htb]
  \centering
  \includegraphics[scale=0.60]{./Figures/figure.pdf}
  \caption{SCALE/TRITON-NEWT Model of BWR Assembly in Order to Show and Example of a Figure and a Multi-line Caption}   
  \label{fig:amdahl}
\end{figure}

When importing figures or any graphical image please verify two things:
\vspace{-0.65cm} % THE DISTANCE BETWEEN THE ":" AND THE FIRST LINE OF THE LIST MAY VARY DEPENDING ON THE TEXT LENGTH, CHANGE THIS VALUE DEPENDING ON YOUR NEEDS.
\begin{itemize} \itemsep1pt \parskip0pt \parsep0pt
\item Any number, text or symbol is in Times font and is not smaller than 
  10-point after reduction to the actual window in your paper
\item That it can be translated into PDF
\end{itemize}

Equations, such as Eq. (\ref{sample_equation}), should be centered and 
sequentially numbered to the flush right of the formula.

\begin{equation}
  \label{sample_equation}
  \mathrm{Speedup}=\frac{1}{\frac{f}{p}+(1-f)}
\end{equation}

The continuation of a paragraph after an equation should not be indented.  
All paragraphs, as well as section or subsection headings, are separated by 
just one single empty line.

\subsubsection{Sub-subsection level and lower: only first character uppercase}

See Table \ref{table:example} for a sample table.  The ``tabls'' package is
recommended for improved row and column spacing.  Notice the caption appears 
above the table by setting the \verb!\caption! command immediately 
after the \verb!\begin{table}!. Tables are numbered in Roman 
numerals, with the caption centered above the table, in \textbf{boldface}.  
Triple-space before and after the table.

\begin{table}[!htb]
  \centering
  \caption{\bf Parallel Performance for the Sample Problem}
  \label{table:example} 
  \begin{tabular}{|c|c|c|c|} \hline 
   Number of & Wall-Clock & Speedup & Efficiency \\
   Processors & Time$^{a}$ (min) & (T$_{s}$/T$_{p}$) & (\%) \\ \hline
    \ 1 &  100.0 & \ ---    & ---  \\ \hline
    \ 2 &   52.6 & \ 1.9    & 95.0 \\ \hline 
  \end{tabular}
\end{table}

\section{CONCLUSIONS}

Present your summary and conclusions here.

\section*{NOMENCLATURE}

If variables are extensively used in the text, a Nomenclature section would be helpful to the readers.

\section*{ACKNOWLEDGEMENTS}

Acknowledge the help of colleagues and sources of funding, as appropriate.

\textbf{As an example:} The format for this template was adapted from the \LaTeX template for the PHYSOR-2018 conference posted and available on the Internet and 
most of the \LaTeX\ format definitions contained in this were already defined. The 
M\&C 2019 organizing committee deeply thank the PHYSOR-2018 technical committee 
for this great support.

% You can enter a bibliography into the document using the following format, or use the 
% bibliography style file "mandc.bst" found in the template directory.  You can use the bibliography style file
% by replacing the current bibliography block with:
% \setlength{\baselineskip}{12pt}
% \bibliographystyle{mandc}
% \bibliography{mandc}

\setlength{\baselineskip}{12pt}
\begin{thebibliography}{300}
\bibitem{journal} B. Author(s), ``Title,'' \emph{Journal Name in Italic}, 
  \textbf{Volume in Bold}, pp. 34-89 (19xx).
\bibitem{proc_paper} C. D. Author(s), ``Article Title,'' \emph{Proceedings of
  Meeting in Italic}, Location, Dates of Meeting, Vol. n, pp. 134-156 
  (19xx).
\bibitem{book} E. F. Author, \emph{Book Title in Italic}, Publisher, City \&
  Country (19xx). 
\bibitem{website} ``Spallation Neutron Source: The next-generation 
  neutron-scattering facility in the United States,'' 
  \url{http://www.sns.gov/documentation/sns\_brochure.pdf} (2002).
\end{thebibliography}

\appendix
\gdef\thesection{APPENDIX \Alph{section}}
\section{Sample Appendix 1}
\label{app:a}
If necessary, include Appendices numbered in upper case alphabetical order. This is \ref{app:a}. 



\end{document}
