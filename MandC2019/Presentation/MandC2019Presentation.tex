\pdfminorversion=4
\documentclass[xcolor={usenames,dvipsnames,svgnames,table},10pt]{beamer}
\geometry{paperwidth=140mm,paperheight=105mm}

\mode<presentation>
\usetheme{Madrid}

\usecolortheme[RGB={80,0,0}]{structure}
\useoutertheme[subsection=false]{miniframes}
\useinnertheme{default}

% hide navigation controlls
\setbeamertemplate{navigation symbols}{}

\setbeamercolor{normal text}{fg=black}
\setbeamercovered{dynamic}
\beamertemplatetransparentcovereddynamicmedium
%\usepackage{chronology}
\setbeamertemplate{caption}[numbered]

\definecolor{Maroon}{RGB}{80,0,0}
\definecolor{BurntOrange}{RGB}{204,85,0}

% load macros and prevent authblk from loading
\input{common/macros.tex}
\dontusepackage{authblk}

% load packages, settings and definitions
\input{common/packages.tex}
\input{common/settings.tex}
\input{common/definitions.tex}

% nicer item settings
\setlist[1]{nolistsep,label=\(\textcolor{Maroon}{\blacksquare}\)}
\setlist[2]{nolistsep,label=\(\textcolor{Maroon}{\bullet}\)}

\newcommand {\mathsym}[1]{{}}
\newcommand {\unicode}{{}}
\newcommand{\om}{\boldsymbol{\Omega}}
\newcommand{\etal}{{\it et al.\,}}
\newcommand{\vr}{\vec{r}}
\newcommand{\vo}{\vec{\Omega}}
\newcolumntype{L}{>{\centering\arraybackslash}m{3cm}}
\newcommand{\tcr}[1]{\textcolor{red}{#1}}

\setenumerate[1]{
	label=\protect\usebeamerfont{enumerate item}%
	\protect\usebeamercolor[fg]{enumerate item}%
	\insertenumlabel.
}

%%%%%%%%%%%%%%%%%%%%%%%%%%%%%%%%%%%%%%%%%%%%%%%
%%% edit to fit your document

% set up pdf support and indexing
\hypersetup{
    pdftitle={<Title>},
    pdfauthor={<author>},
    pdfsubject={<subject>},
    pdfkeywords={<keywords>},
}

\title[Approaches to Load Balancing]{Approaches to Load Balancing Massively Parallel Transport Sweeps on Unstructured Grids}
\author[Ghaddar]{Tarek Habib Ghaddar, Jean C. Ragusa}
\institute[Texas A\&M]{Department of Nuclear Engineering \\ Texas A\&M University}
\date[August 27, 2019]

\begin{document}

% title page, do not edit
{
\setbeamertemplate{headline}[default] 
\begin{frame}
\vspace{-1.1cm}
	\begin{figure}[t]
		\centering
			\includegraphics[width=.25\textwidth]{images/seal.png}
	\end{figure}
\vspace{-0.75cm}
\titlepage
\end{frame}
}

\begin{frame}
\tableofcontents
\end{frame}

%%%%%%%%%%%%%%%%%%%%%%%%%%%%%%%%%%%%%%%%%%%%%%%%%%%%%%%%%%%%%%%%%%
%%%%%%%%%%%%%%%%%%%%%%%%%%%%%%%%%%%%%%%%%%%%%%%%%%%%%%%%%%%%%%%%%%
%%%%%%%%%%%%%%%%%%%%%%%%%%%%%%%%%%%%%%%%%%%%%%%%%%%%%%%%%%%%%%%%%%
\section{Introduction}
\subsection{}
%%%%%%%%%%%%%%%%%%%%%%%%%%%%%%%%%%%%%%%%%%%%%%%%%%%%%%%%%%%%%%%%%%
%%%%%%%%%%%%%%%%%%%%%%%%%%%%%%%%%%%%%%%%%%%%%%%%%%%%%%%%%%%%%%%%%%
%%%%%%%%%%%%%%%%%%%%%%%%%%%%%%%%%%%%%%%%%%%%%%%%%%%%%%%%%%%%%%%%%%

\begin{frame}[t]\frametitle{Introduction}
\begin{block}{}
\begin{itemize}
	\item Massively parallel transport sweeps have been shown to scale up to 1.5 million processes on logically cartesian grids.
	\item Structured meshes are somewhat limiting when attempting to model more complex problems and experiments.
	\item Unstructured meshes allow us to model realistic problems, but introduce unbalanced partitions. 
	\item PDT (Texas A\&M's massively parallel transport code) introduced two load balancing algorithms that repartition the mesh in order to obtain a roughly equivalent amount of cells per processor. 
\end{itemize}
\end{block}
\end{frame}

\begin{frame}[t]\frametitle{Transport Sweeps}
\begin{block}{}
\begin{equation}
\vo_m \cdot \vec\nabla \psi_m^{(l+1)}(\vr) + \Sigma_t \psi_m^{(l+1)}(\vr) = q_m^{(l)}(\vr),
\label{iteration}
\end{equation}
\begin{itemize}
\item The domain is meshed, allowing one cell at a time to be solved.
\item The solution across a cell interface is connected based on an upwind approach, allowing cells to be solved one at a time.
\end{itemize}
\end{block}
\centering
\includegraphics[scale = 0.15]{../../figures/UnstructureMesh.pdf}
\includegraphics[scale = 0.15]{../../figures/StructuredMesh.pdf}
\end{frame}

\begin{frame}[t]\frametitle{Parallel Transport Sweeps}

\begin{block}{A parallel sweep algorithm is defined by three properties:}
\begin{itemize}
\item partitioning: dividing the domain among available processors,
\item aggregation: grouping cells, directions, and energy groups into tasks,
\item scheduling: choosing which task to execute if more than one is available.
\end{itemize}
\end{block}
\centering
\includegraphics[scale=0.25]{../../figures/sweep_scale2b.png}
\end{frame}

%\begin{frame}[t]\frametitle{Tie Breaking in PDT}
%
%\begin{block}{If two or more tasks reach a processor at the same time, PDT employs a tie breaking strategy:}
%
%\begin{enumerate}
%	\item The task with the greater depth-of-graph remaining (simply, more work remaining) goes first.
%	\item If the depth-of-graph remaining is tied, the task with $\Omega_x > 0$ wins.
%	\item If multiple tasks have $\Omega_x > 0$, then the task with $\Omega_y > 0$ wins.
%	\item If multiple tasks have $\Omega_y > 0$, then the task with $\Omega_z > 0$ wins.
%\end{enumerate}
%\end{block}
%\end{frame}

\begin{frame}[t]\frametitle{Unstructured Meshing in PDT}
  \begin{block}{}
    \begin{itemize}
      \item PDT using unstructured meshes has been a priority since early 2014. 
      \item Unstructured meshes allow for simulation of a wider and more general variety of problems.
      \item Three unstructured mesh types are supported in PDT:
        \begin{itemize}
          \item Triangle (2D and 2D extruded triangular/prismatic meshes).
          \item Spiderweb (2D and 2D extruded prismatic meshes).
          \item Cubit/OpenFOAM (fully unstructured 3D meshes).
        \end{itemize}
    \end{itemize}
  \end{block}
  \centering
  \includegraphics[scale=0.10]{../../figures/im1_228.png}
\end{frame}

\begin{frame}[t]\frametitle{Partitioning in PDT}
\begin{block}{Partitioning for Unstructured Meshes}
\begin{itemize}
\item ``Cut lines" in 2D (cut planes for 3D) are used to slice through the mesh in the $x$, $y$, and $z$ dimensions.
\item The cut planes form brick partitions, called subsets, that have unstructured meshes inside of them. 
\item The subsets are distributed amongst the processor domain.
\end{itemize}
\end{block}
  \begin{block}{PWLD}
     Using Piece-Wise Linear Discontinuous finite element basis functions, we can solve the transport equation for arbitrary and degenerate polyhedra.
  \end{block}
  \centering
  \includegraphics[scale=0.25]{../../figures/deg_square_PWLD1_contour_b5.png}
\end{frame}

\begin{frame}[t]\frametitle{Partitioning in PDT}
  \begin{columns}
    \column{0.5\textwidth}
    \begin{minipage}[c][0.4\textheight][c]{\linewidth}
      \includegraphics[scale=0.14,trim={0.95in 0.64in 0.35in 0.44in},clip]{../../figures/im1_cubit_slice.png}
    \end{minipage}
    \begin{minipage}[c][0.4\textheight][c]{\linewidth}
      \includegraphics[scale=0.05]{../../figures/FuelAndGuideZoom.png}
    \end{minipage}
    \column{0.5\textwidth}
    \begin{minipage}[c][0.4\textheight][c]{\linewidth}
      \begin{block}{}
        PWLD allows us to solve these partitioned arbitrary and degenerate meshes.
      \end{block}
    \end{minipage}
    \begin{minipage}[c][0.4\textheight][c]{\linewidth}
      \includegraphics[scale=0.15]{../../figures/spiderweb4.png}
    \end{minipage}
  \end{columns}
\end{frame}

%\begin{frame}[t]\frametitle{Partitioning Example}
%\centering
%\includegraphics[scale=0.25,trim={0.95in 0.64in 0.35in 0.44in},clip]{../../figures/unbalanced_pins_regular_10.png}
%\end{frame}

\section{Load Balancing}
\subsection{}

\begin{frame}[t]\frametitle{Load Balancing By Dimension}
\begin{block}{}
  \begin{itemize}
    \item The load balancing by dimension algorithm expands upon load balancing work presented at M\&C 2017.
    \item We load balance by moving partitions into mesh dense areas in order to redistribute the workload more evenly across processors.
    \item The load balancing algorithms rely on a few metrics to assist with the movement of cut lines.
  \end{itemize}		
\end{block}
\end{frame}

\begin{frame}[t]\frametitle{Redistribution}
\begin{figure}[H]
\centering
\includegraphics[scale=0.37]{../../figures/spiderweb_redistribute_before.pdf}
\includegraphics[scale=0.37]{../../figures/spiderweb_redistribute_after.pdf}
\caption{The use of the CDF of triangles per column to redistribute the cut planes in X.}
\label{redistribute}
\end{figure}
\end{frame}


\begin{frame}[t]\frametitle{Load Balancing Metrics}
\begin{equation}
f =\frac{\underset{ijk}{\text{max}}(N_{ijk})}{\frac{N_{tot}}{I\cdot J\cdot K}},
\label{metric_def}
\end{equation}
\begin{align}
f_{K} &= \underset{k}{\text{max}}[\sum_{i,j} N_{ijk}]/\frac{N_{tot}}{K}, \label{f_d1} \\
f_{I,k} &= \Big(\underset{i}{\text{max}}[\sum_{j} N_{ijk,k}]/\frac{N_{k}}{I}\Big), \label{f_d2}\\
f_{J,i,k} &= \Big( \underset{j}{\text{max}}[ N_{ijk,k,i}]/\frac{N_{k,i}}{J} \Big) . \label{f_j}
\end{align}
\end{frame}

\begin{frame}[t]\frametitle{Original Load Balancing Algorithm (2D Only)}
\begin{algorithm}[H]
\label{initial_algorithm}
\begin{algorithmic}

\WHILE{$f > \text{tol}_{\text{subset}}$}
  \IF {$f_I > \text{tol}_{\text{col}}$}
    \STATE Redistribute the X cut planes.
  \ENDIF
  \IF {$f_J > \text{tol}_{\text{row}}$}
  	\STATE Redistribute the Y cut planes.
  \ENDIF
\ENDWHILE
\end{algorithmic}
\end{algorithm}
\end{frame}

\begin{frame}[t]\frametitle{Theoretical Motivation for LBD}
  \begin{block}{}
  \begin{itemize}
    \item Consider simple 2D layout with $M$ unaligned subsets of high mesh density that each have $N$ cells.
    \item There are $M^2$ subsets, but only M have much work.
    \item Load Imbalance Factor $= \frac{N}{(MN+C)/M^2} \xrightarrow{N\to \infty} \frac{N}{N/M} = M$
  \end{itemize}
  \end{block}
  \begin{center}
    \includegraphics[scale=0.2]{../../figures/theoretical_plot.png}
  \end{center}
\end{frame}

\begin{frame}[t]\frametitle{Load Balancing By Dimension Algorithm}
\begin{algorithm}[H]
\label{lbd}
\begin{algorithmic}

  \WHILE {$f_{K} > \text{tol}_{\text{K}}$}
    \STATE Redistribute the Z cut planes.
  \ENDWHILE  
  
  \FOR {$k$ in $K$}
    \WHILE {$f_{I,k} > \text{tol}_{\text{I}}$}
      \STATE Redistribute the X cut planes within each Z layer. 
    \ENDWHILE
  \ENDFOR
  
  \FOR{$k$ in $K$}
    \FOR{$i$ in $I$}
      \WHILE {$f_{J,i,k} > \text{tol}_{\text{J}}$ }
        \STATE Redistribute the Y cut planes within each column within each Z layer. 
      \ENDWHILE
    \ENDFOR
  \ENDFOR
  
  \STATE Calculate $f$.
\end{algorithmic}
\end{algorithm}
\end{frame}


%\begin{frame}[t]\frametitle{Paramtric Study Conclusions}
%  \begin{block}{}
%  \begin{itemize}
%    \item The more uniformly refined your mesh, the more inherently balanced it is.
%    \item With the exception of a few outliers, the load balancing by dimension algorithm was an improvement over the original load balancing algorithm.
%    \item The metric improved by a max of 76.9\% and a mean of 21.7\% with the load balancing by dimension algorithm over the original load balancing algorithm. 
%  \end{itemize}
%  \end{block}
%\end{frame}

\begin{frame}[t]\frametitle{Testing Load Balancing}
\begin{block}{}
\begin{itemize}
  \item This mesh was run through the load balancing and load balancing by dimension algorithm.
  \item The number of subsets ranged from 2 to 7 in both dimensions.
\end{itemize}
\end{block}
  \centering
  \includegraphics[scale=0.17,trim={2cm 0cm 1cm 1.3cm},clip]{../../figures/unbalanced_pins_refined.png}
\end{frame}

\begin{frame}[t]\frametitle{Results of Load Balancing By Dimension}
\centering
\includegraphics[scale=0.6]{../../figures/lbd_results.pdf}
Minimum LBD metric: 1.07692, Max LBD metric: 1.33636
\end{frame}

\begin{frame}[t]\frametitle{Consequences of Load Balancing By Dimension}
  \begin{block}{}
  \begin{itemize}
    \item Perfect load balance in some cases will come at the cost of optimal sweeping.
    \item Time to solution is the most pertinent parameter, and if keeping a more optimal sweeping grid means a less balanced problem, then so be it.
    \item With imbalanced partitions it is harder to characterize the idle time, or the time where a processor has no work to do, and thus obtain an accurate stage count.
    \item A time-to-solution estimator must be built to more accurately predict sweep time.
  \end{itemize}
  \end{block}
\end{frame}

\begin{frame}[t]\frametitle{3D LBD, $f = 1.0199$}
\centering
\includegraphics[trim={0cm 1cm 0cm 3cm},clip,scale=0.23]{../../figures/im1_foam_448.png}
\end{frame}

\begin{frame}[t]\frametitle{Sweep on Regular Grid with 3 Angle Sets}
    \centering
	\animategraphics[loop,controls,width=0.7\linewidth]{10}{../../figures/sweeps_png/sweep_regular_20x20_as3_dog/sweep_regular_20x20_as3_dog_}{1}{48}
	%\href{run:figures/sweep_figs/sweeps_png/sweep_regular_20x20_as3_dog/animation.gif}{Animation.gif}
\end{frame}

\begin{frame}[t]\frametitle{Sweep on LBD Grid with 3 Angle Sets}
   \centering
	\animategraphics[loop,controls,width=0.7\linewidth]{10}{../../figures/sweeps_png/sweep_random_20x20_as3_dog/sweep_random_20x20_as3_dog_}{1}{101}
	%\href{run:figures/sweep_figs/sweeps_png/sweep_random_20x20_as3_dog/animation.gif}{Animation.gif}
\end{frame}


\begin{frame}[t]\frametitle{Sweep on Worst Grid with 1 Angle Set}
    \centering
	\animategraphics[loop,controls,width=0.7\linewidth]{10}{../../figures/sweeps_png/sweep_worst_20x20_as1_dog/sweep_worst_20x20_as1_dog_}{1}{230}
	%\href{run:figures/sweep_figs/sweeps_png/sweep_worst_20x20_as1_dog/animation.gif}{Animation.gif}
\end{frame}

\section{Ongoing Work}
\subsection{}
%%%%%%%%%%%%%%%%%%%%%%%%%%%%%%%%%%%%%%%%%%%%%%%%%%%%%%%%%%%%%%%%%%%%%%%%%%%%%%%%%%%%%%%
%%%%%%%%%%%%%%%%%%%%%%%%%%%%%%%%%%%%%%%%%%%%%%%%%%%%%%%%%%%%%%%%%%%%%%%%%%%%%%%%%%%%%%%
%%%%%%%%%%%%%%%%%%%%%%%%%%%%%%%%%%%%%%%%%%%%%%%%%%%%%%%%%%%%%%%%%%%%%%%%%%%%%%%%%%%%%%%
%\section{Time-To-Solution Estimator}
%\subsection{}
%%%%%%%%%%%%%%%%%%%%%%%%%%%%%%%%%%%%%%%%%%%%%%%%%%%%%%%%%%%%%%%%%%%%%%%%%%%%%%%%%%%%%%%%
%%%%%%%%%%%%%%%%%%%%%%%%%%%%%%%%%%%%%%%%%%%%%%%%%%%%%%%%%%%%%%%%%%%%%%%%%%%%%%%%%%%%%%%%
%%%%%%%%%%%%%%%%%%%%%%%%%%%%%%%%%%%%%%%%%%%%%%%%%%%%%%%%%%%%%%%%%%%%%%%%%%%%%%%%%%%%%%%%
%
%\begin{frame}[t]\frametitle{Overview}
%\begin{block}{}
%\begin{itemize}
%	\item We need to optimize the cut plane location not for balance, but for the best possible sweep time.
%	\item We must build a time-to-solution estimator that calculates the time to solution for a given cut line partitioning and mesh cell density.
%	\item The time to solution estimator will be fed into an optimizing function that minimizes the time to solution. The cut planes corresponding to the minimum time to solution are the optimal partitioning scheme.
%\end{itemize}
%\end{block}
%\begin{block}{IMPORTANT}
%The time-to-solution estimator is a graph-based method that uses graph algorithms that rely on an acyclic graph. The partitioning schemes used CANNOT induce cycles.
%\end{block}
%\end{frame}
%
\begin{frame}[t]\frametitle{Time To Solution Estimator}
\begin{block}{}
\begin{enumerate}
    \item Given a partitioning scheme, build an adjacency matrix.
    \item From the adjacency matrix, build Directed Acyclic Graphs (DAGs), one for each quadrant/octant.
    \item Weight each DAG's edges based on the solve and communication time of each subset to its neighbors.
    \item Adjust the weights of each graph to reflect a universal timescale.
    \item Adjust the weights of each graph to reflect sweep conflicts between octants.
    \item Obtain the time-to-solution.
\end{enumerate}
\end{block}
\end{frame}
%
%\begin{frame}[t]\frametitle{Building the Adjacency Matrix}
%\begin{block}{}
%\begin{itemize}
%  \item The adjacency matrix provides the crucial connectivity information necessary to build the Task Dependence Graphs (TDG's) in the time-to-solution estimator.
%\end{itemize}
%\end{block}
%\begin{minipage}{0.49\textwidth}
%\centering
%\includegraphics[scale=0.4]{../../figures/base_adjacency_layout.pdf}
%\end{minipage}
%\begin{minipage}{0.49\textwidth}
%\centering
%$\begin{pmatrix}
%0 & 1 & 1  & 0  \\
%1 & 0 & 0 & 1 \\
%1 & 0 & 0 & 1 \\ 
%0 & 1 & 1 & 0 \\
%\end{pmatrix}$
%\end{minipage}
%\end{frame}
%
%\begin{frame}[t]\frametitle{Building the Task Dependence Graphs}
%\begin{block}{}
%\begin{itemize}
%    \item Each quadrant/octant gets its own TDG.
%    \item The opposing quadrant/octant pairs have opposing sweep orderings. 
%    \item Each set of quadrant/octant pairs uses a different adjacency matrix. 
%    \item Python package \textbf{networkx} is used for all graph operations.
%\end{itemize}
%\end{block}
%\centering
%\includegraphics[scale=0.45]{../../figures/quadrant_layout.pdf}
%\end{frame}
%
%\begin{frame}[t]\frametitle{Quadrants 0 and 3}
%\begin{block}{}
%\begin{itemize}
%    \item We use the upper triangular portion of the adjacency matrix for quadrant 0.
%    \item We use the lower triangular portion of the adjacency matrix for quadrant 3.
%\end{itemize}
%\end{block}
%\begin{minipage}{0.49\textwidth}
%  \centering
%  \includegraphics[scale=0.35]{../../figures/q0_preweight.pdf}
%\end{minipage}
%\begin{minipage}{0.49\textwidth}
%  \centering
%  \includegraphics[scale=0.35]{../../figures/q3_preweight.pdf}
%\end{minipage}
%\end{frame}
%
%\begin{frame}[t]\frametitle{Quadrants 1 and 2}
%  \begin{block}{}
%    \begin{itemize}
%      \item Before building the graphs for these opposing quadrants, we have to alter the adjacency matrix slightly.
%      \item The subsets get renumbered in a manner where subset 1 becomes subset 0, subset 0 becomes subset 1, etc. In short the column numbering is flipped, and a ``flipped'' adjacency matrix is built. 
%    \end{itemize}
%  \end{block}
%  \begin{minipage}{0.49\textwidth}
%    \centering
%    \includegraphics[scale=0.38]{../../figures/base_adjacency_layout.pdf}
%  \end{minipage}
%  \begin{minipage}{0.49\textwidth}
%    \centering
%    \includegraphics[scale=0.38]{../../figures/flipped_subset_layout.pdf}
%  \end{minipage}
%\end{frame}
%
%\begin{frame}[t]\frametitle{Quadrants 1 and 2}
%  \begin{block}{}
%    \begin{itemize}
%      \item We use the upper triangular portion of the ``flipped'' adjacency matrix for quadrant 1.
%      \item We use the lower triangular portion of the ``flipped'' adjacency matrix for quadrant 2.
%    \end{itemize}
%  \end{block}
%  \begin{minipage}{0.49\textwidth}
%    \centering
%    \includegraphics[scale=0.35]{../../figures/q1_preweight.pdf}
%  \end{minipage}
%  \begin{minipage}{0.49\textwidth}
%    \centering
%    \includegraphics[scale=0.35]{../../figures/q2_preweight.pdf}
%  \end{minipage}
%\end{frame}
%
\begin{frame}[t]\frametitle{Weighting the TDGs}
  \begin{block}{}
    \begin{itemize}
    	\item In order to weight the graph properly, we need to have a mesh density function that we can calculate the cells per subset from.
    	\item The weight of the edge between two nodes (subsets) in the graph represents the solve and communication time of the base node.
    \end{itemize}
  \end{block}
  \begin{block}{}
    \begin{align}
    \text{cells per subset} &= \int_{\tcr{x_i}}^{\tcr{x_{i+1}}} \int_{\tcr{y_j}}^{\tcr{y_{j+1}}} \int_{\tcr{z_k}}^{\tcr{z_{k+1}}} \text{mesh density } dx dy dz \\
    \label{weight}
      \text{weight} &= N_c\cdot (T_c+ A_m\cdot(T_m + T_g)) + N_{b}\cdot A_m\cdot T_{\text{comm}} + \text{latency}\cdot M_L \\
      N_u &= \text{number of cells} \\
    N_b &\approx(\text{num boundary cells})\cdot \text{boundary unknowns per cell}
    \end{align}
  \end{block}
\end{frame}
%
%\begin{frame}[t]\frametitle{Weighting the TDGs}
%  \begin{minipage}{0.49\textwidth}
%    \centering
%    \includegraphics[scale=0.32]{../../figures/q1_postweight.pdf}
%  \end{minipage}
%  \begin{minipage}{0.49\textwidth}
%    \centering
%    \includegraphics[scale=0.32]{../../figures/q3_postweight.pdf}
%  \end{minipage}
%  \begin{minipage}{0.49\textwidth}
%    \centering
%    \includegraphics[scale=0.32]{../../figures/q0_postweight.pdf}
%  \end{minipage}
%  \begin{minipage}{0.49\textwidth}
%    \centering
%    \includegraphics[scale=0.32]{../../figures/q2_postweight.pdf}
%  \end{minipage}
% 
%\end{frame}
%
%\begin{frame}[t]\frametitle{Adjusting for Multiple Angles}
% \begin{minipage}{0.49\textwidth}
%    \centering
%    \includegraphics[scale=0.32]{../../figures/q1_postpipeline.pdf}
%  \end{minipage}
%  \begin{minipage}{0.49\textwidth}
%    \centering
%    \includegraphics[scale=0.32]{../../figures/q3_postpipeline.pdf}
%  \end{minipage}
%  \begin{minipage}{0.49\textwidth}
%    \centering
%    \includegraphics[scale=0.32]{../../figures/q0_postpipeline.pdf}
%  \end{minipage}
%  \begin{minipage}{0.49\textwidth}
%    \centering
%    \includegraphics[scale=0.32]{../../figures/q2_postpipeline.pdf}
%  \end{minipage}
%\end{frame}
%
%\begin{frame}[t]\frametitle{Second Angleset}
% \begin{minipage}{0.49\textwidth}
%    \centering
%    \includegraphics[scale=0.32]{../../figures/q5_postpipeline.pdf}
%  \end{minipage}
%  \begin{minipage}{0.49\textwidth}
%    \centering
%    \includegraphics[scale=0.32]{../../figures/q7_postpipeline.pdf}
%  \end{minipage}
%  \begin{minipage}{0.49\textwidth}
%    \centering
%    \includegraphics[scale=0.32]{../../figures/q4_postpipeline.pdf}
%  \end{minipage}
%  \begin{minipage}{0.49\textwidth}
%    \centering
%    \includegraphics[scale=0.32]{../../figures/q6_postpipeline.pdf}
%  \end{minipage}
%\end{frame}
%
\begin{frame}[t]\frametitle{Universal Edge Weighting}
  \begin{block}{}
    \begin{itemize}
      \item Once the outgoing edges of each node are independently weighted, the next step is to have our graphs weighted on a universal time scale. 
      \item To do this, we calculate the sum of the longest weighted path to each node, and set all incoming edges to each node to this value.
      \item To find the longest weighted path, the graph weights are inverted, and the shortest weighted path is found. The original edge weights are used to calculate the value of the longest weighted path.
      \item The resulting graph is weighted so that the incoming edge to each node represents the time that node is ready to solve.
    \end{itemize}
  \end{block}
\end{frame}
%\begin{frame}[t]\frametitle{Universal Edge Weighting}
% \begin{minipage}{0.49\textwidth}
%    \centering
%    \includegraphics[scale=0.32]{../../figures/q1_postuniversal.pdf}
%  \end{minipage}
%  \begin{minipage}{0.49\textwidth}
%    \centering
%    \includegraphics[scale=0.32]{../../figures/q3_postuniversal.pdf}
%  \end{minipage}
%  \begin{minipage}{0.49\textwidth}
%    \centering
%    \includegraphics[scale=0.32]{../../figures/q0_postuniversal.pdf}
%  \end{minipage}
%  \begin{minipage}{0.49\textwidth}
%    \centering
%    \includegraphics[scale=0.32]{../../figures/q2_postuniversal.pdf}
%  \end{minipage}
%\end{frame}
%
%\begin{frame}[t]\frametitle{Second Angleset}
% \begin{minipage}{0.49\textwidth}
%    \centering
%    \includegraphics[scale=0.32]{../../figures/q5_postuniversal.pdf}
%  \end{minipage}
%  \begin{minipage}{0.49\textwidth}
%    \centering
%    \includegraphics[scale=0.32]{../../figures/q7_postuniversal.pdf}
%  \end{minipage}
%  \begin{minipage}{0.49\textwidth}
%    \centering
%    \includegraphics[scale=0.32]{../../figures/q4_postuniversal.pdf}
%  \end{minipage}
%  \begin{minipage}{0.49\textwidth}
%    \centering
%    \includegraphics[scale=0.32]{../../figures/q6_postuniversal.pdf}
%  \end{minipage}
%\end{frame}
%
\begin{frame}[t]\frametitle{Conflict Detection and Resolution}
  \begin{block}{}
    \begin{itemize}
      \item Best summarized as a ``marching" process. 
      \item Starting at time $t=0$, find the first interaction (recalling that the incoming weight to a node reflects the time it is ready to solve).
      \item If at time $t$, multiple TDGs are solving the same node, this means there is a conflict. 
      \item  ``Losing" TDGs modify their downstream weights according to how long they are delayed.
    \end{itemize}
  \end{block}
\end{frame}
%
\begin{frame}[t]\frametitle{First Come First Serve Conflict Resolution}
\begin{block}{}
\begin{itemize}
	\item The first octant to arrive to a node will begin solving it, and the remaining octants will incur a delay (if applicable).
	\item The delay is reflected in each remaining TDG by adding the corresponding delay as a weight to the applicable edge and applicable downstream edges.
  \item If two octants arrive to a node at the same time, the octant with the greater remaining depth-of-graph and priority octant wins the tie.
\end{itemize}
\end{block}
  \begin{block}{Unweighted vs. Weighted Depth-of-Graph Remaining}
    \begin{itemize}
      \item There are two options for the depth-of-graph remaining. Unweighted does not take the weights into account, and weighted looks at the final time for the conflicting graphs on the universal time scale.
    \end{itemize}
  \end{block}
\end{frame}
%
%\begin{frame}[t]\frametitle{$t=1$: Graphs 0,3 and Graphs 1,2 in Conflict}
% \begin{minipage}{0.49\textwidth}
%    \centering
%    \includegraphics[scale=0.32]{../../figures/graph_1_0_1.pdf}
%  \end{minipage}
%  \begin{minipage}{0.49\textwidth}
%    \centering
%    \includegraphics[scale=0.32]{../../figures/graph_1_0_3.pdf}
%  \end{minipage}
%  \begin{minipage}{0.49\textwidth}
%    \centering
%    \includegraphics[scale=0.32]{../../figures/graph_1_0_0.pdf}
%  \end{minipage}
%  \begin{minipage}{0.49\textwidth}
%    \centering
%    \includegraphics[scale=0.32]{../../figures/graph_1_0_2.pdf}
%  \end{minipage}
%\end{frame}
%
%\begin{frame}[t]\frametitle{$t=2$: Graphs 0 and 1 Won, Still in Conflict}
% \begin{minipage}{0.49\textwidth}
%    \centering
%    \includegraphics[scale=0.32]{../../figures/graph_2_1_1.pdf}
%  \end{minipage}
%  \begin{minipage}{0.49\textwidth}
%    \centering
%    \includegraphics[scale=0.32]{../../figures/graph_2_1_3.pdf}
%  \end{minipage}
%  \begin{minipage}{0.49\textwidth}
%    \centering
%    \includegraphics[scale=0.32]{../../figures/graph_2_1_0.pdf}
%  \end{minipage}
%  \begin{minipage}{0.49\textwidth}
%    \centering
%    \includegraphics[scale=0.32]{../../figures/graph_2_1_2.pdf}
%  \end{minipage}
%\end{frame}
%
%\begin{frame}[t]\frametitle{$t=3$: Graphs 2 and 3 Won, No More Conflict}
% \begin{minipage}{0.49\textwidth}
%    \centering
%    \includegraphics[scale=0.32]{../../figures/graph_3_2_1.pdf}
%  \end{minipage}
%  \begin{minipage}{0.49\textwidth}
%    \centering
%    \includegraphics[scale=0.32]{../../figures/graph_3_2_3.pdf}
%  \end{minipage}
%  \begin{minipage}{0.49\textwidth}
%    \centering
%    \includegraphics[scale=0.32]{../../figures/graph_3_2_0.pdf}
%  \end{minipage}
%  \begin{minipage}{0.49\textwidth}
%    \centering
%    \includegraphics[scale=0.32]{../../figures/graph_3_2_2.pdf}
%  \end{minipage}
%\end{frame}
%
%\begin{frame}[t]\frametitle{$t=4$, Done Sweeping!}
% \begin{minipage}{0.49\textwidth}
%    \centering
%    \includegraphics[scale=0.32]{../../figures/graph_4_3_1.pdf}
%  \end{minipage}
%  \begin{minipage}{0.49\textwidth}
%    \centering
%    \includegraphics[scale=0.32]{../../figures/graph_4_3_3.pdf}
%  \end{minipage}
%  \begin{minipage}{0.49\textwidth}
%    \centering
%    \includegraphics[scale=0.32]{../../figures/graph_4_3_0.pdf}
%  \end{minipage}
%  \begin{minipage}{0.49\textwidth}
%    \centering
%    \includegraphics[scale=0.32]{../../figures/graph_4_3_2.pdf}
%  \end{minipage}
%\end{frame}
%
%
%\begin{frame}[t]\frametitle{Synthetic Probable-Worst Case with 100 Points/Subset}
%   \begin{minipage}{0.49\textwidth}
%    \centering
%    \includegraphics[scale=0.35]{../../figures/synthetic_lbd_cuts.pdf}
%    \begin{block}{}
%    \centering
%      $t = 1.335$
%    \end{block}
%  \end{minipage}
%  \begin{minipage}{0.49\textwidth}
%    \centering
%    \includegraphics[scale=0.35]{../../figures/synthetic_opt_cuts.pdf}
%    \begin{block}{}
%    \centering
%    $t = 0.756$
%    \end{block}
%  \end{minipage}
%  \vspace{1cm}
%  \centering
%    Improvement: $\frac{0.756}{1.335} = 0.566$
%\end{frame}
%
\begin{frame}[t]\frametitle{Synthetic Probable-Worst Case with 10 Points/Subset}
   \begin{minipage}{0.49\textwidth}
    \centering
    \includegraphics[scale=0.35]{../../figures/synthetic_lbd_cuts_lighter.pdf}
    \begin{block}{}
    \centering
      $t = 0.145$
    \end{block}
  \end{minipage}
  \begin{minipage}{0.49\textwidth}
    \centering
    \includegraphics[scale=0.35]{../../figures/synthetic_opt_cuts_lighter.pdf}
    \begin{block}{}
    \centering
    $ t = 0.080$
    \end{block}
  \end{minipage}
  \vspace{1cm}
  \centering
    Improvement: $\frac{0.080}{0.145} = 0.552$
\end{frame}

\begin{frame}[t]\frametitle{Synthetic Probable-Worst Case with 1000 Points/Subset}
   \begin{minipage}{0.49\textwidth}
    \centering
    \includegraphics[scale=0.35]{../../figures/synthetic_lbd_cuts_heavy.png}
    \begin{block}{}
    \centering
    $t = 12.499$
    \end{block}
  \end{minipage}
  \begin{minipage}{0.49\textwidth}
    \centering
    \includegraphics[scale=0.35]{../../figures/synthetic_opt_cuts_heavy.png}
    \begin{block}{}
    \centering
    $t = 7.298$
    \end{block}
  \end{minipage}
  \vspace{1cm}
  \centering
    Improvement: $\frac{7.289}{12.499} = 0.583$
\end{frame}
%

%
\begin{frame}[t]\frametitle{Level 2 Experiment 42x13}
  \centering
  \includegraphics[scale=0.35,trim={2cm 0cm 1cm 15cm},clip]{../../figures/level2base.png}
\end{frame}

\begin{frame}[t]\frametitle{Level 2 Experiment 42x13}
   \begin{minipage}{0.49\textwidth}
    \centering
    \includegraphics[scale=0.32]{../../figures/lvl2regularcuts.pdf}
    \begin{block}{}
    \centering
      $t = 0.314$
    \end{block}
  \end{minipage}
  \begin{minipage}{0.49\textwidth}
    \centering
    \includegraphics[scale=0.32]{../../figures/lvl2lbcuts.pdf}
    \begin{block}{}
    \centering
      $t = 0.152$
    \end{block}
  \end{minipage}  
\end{frame}
%
\begin{frame}[t]\frametitle{IM1C Experiment with 5 Subsets in Each Dimension}
\centering
\includegraphics[scale=0.15]{../../figures/im1_228.png}
\begin{block}{}
\begin{table}
\begin{tabular}{c|c} 
\bf LBD &\bf Regular   \\ \hline \hline
 0.5723& 0.364
\end{tabular}
\end{table}
\end{block}
\end{frame}
%
%



\begin{frame}[t]\frametitle{Acknowledgements}
\begin{block}{}
A special thank you to the following individuals for their help and support:
\begin{itemize}
\item Drs. Ragusa, Morel, Adams, and Amato
\item Michael Adams, Daryl Hawkins, Timmie Smith
\item Andrew Till
\item The CERT team and fellow grad students (particularly my officemate Ian Halvic, who has dealt with my screaming). 
\end{itemize}
\end{block}
\end{frame}

\end{document}
