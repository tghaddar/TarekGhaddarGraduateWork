%%%%%%%%%%%%%%%%%%%%%%%%%%%%%%%%%%%%%%%%%%%%%%%%%%%
%
%  New template code for TAMU Theses and Dissertations starting Fall 2016.  
%
%
%  Author: Sean Zachary Roberson
%  Version 3.17.09
%  Last Updated: 9/21/2017
%
%%%%%%%%%%%%%%%%%%%%%%%%%%%%%%%%%%%%%%%%%%%%%%%%%%%
%%%%%%%%%%%%%%%%%%%%%%%%%%%%%%%%%%%%%%%%%%%%%%%%%%%%%%%%%%%%%%%%%%%%%
%%                           ABSTRACT 
%%%%%%%%%%%%%%%%%%%%%%%%%%%%%%%%%%%%%%%%%%%%%%%%%%%%%%%%%%%%%%%%%%%%%

\chapter*{ABSTRACT}
\addcontentsline{toc}{chapter}{ABSTRACT} % Needs to be set to part, so the TOC doesnt add 'CHAPTER ' prefix in the TOC.

\pagestyle{plain} % No headers, just page numbers
\pagenumbering{roman} % Roman numerals
\setcounter{page}{2}

\indent The field of radiation transport studies the distribution of radiation throughout a seven-dimensional phase-space consisting of time, space, energy, and direction. Radiation transport is described by the Boltzmann equation and can be solved stochastically or deterministically.

The work presented in this dissertation utilizes the deterministic method known as the transport sweep, a popular technique that has been the subject of a large amount of research.
We specifically focus on the parallel implementations of the transport sweep, and predicting the time it takes to sweep across a structured or unstructured mesh given a set of partitioning parameters, achieved through a time-to-solution estimator, written in Python.
The time-to-solution estimator is tested against PDT, Texas A\&M's massively deterministic transport code.
The time-to-solution estimator's sweep time is within 10\% of PDT's sweep time for the majority of problems tested.

We use the time-to-solution estimator as the objective function in an optimization scheme to attempt to get the partitions that lead to the fastest sweep time for a given problem and partitioning scheme.
Two optimization methods are discussed: using a black box tool (scipy's optimize library) and an intuitive method that relies on placing partitions in mesh locations that does not increase the number of cells (which we chose to name the CDF method).
The time-to-solution estimator proved to not be smooth enough for a black box tool to work, so the geometry based optimization method became the primary method.
The CDF method proved effective for our unbalanced pin test problem, improving the time to solution over previously used partitioning schemes.
For our larger test problem, the optimized partitioning scheme improves the time to solution over one previously used partitioning scheme, but is not as effective relative to other previously used partitioning schemes.
\pagebreak{}
