%%%%%%%%%%%%%%%%%%%%%%%%%%%%%%%%%%%%%%%%%%%%%%%%%%%
%
%  New template code for TAMU Theses and Dissertations starting Fall 2016.  
%
%
%  Author: Sean Zachary Roberson
%  Version 3.17.09
%  Last Updated: 9/21/2017
%
%%%%%%%%%%%%%%%%%%%%%%%%%%%%%%%%%%%%%%%%%%%%%%%%%%%
%%%%%%%%%%%%%%%%%%%%%%%%%%%%%%%%%%%%%%%%%%%%%%%%%%%%%%%%%%%%%%%%%%%%%%
%%                           TIME TO SOLUTION ESTIMATOR CHAPTER
%%%%%%%%%%%%%%%%%%%%%%%%%%%%%%%%%%%%%%%%%%%%%%%%%%%%%%%%%%%%%%%%%%%%%



\chapter{PARTITIONING OPTMIZIATION \label{cha:optimization}}
\section{Method}
With confidence in the time-to-solution estimator, it is used as the objective function in two optimization methods. The first optimization method utilizes scipy's optimize library, and the second method utilizes knowledge of a problem's mesh layout to assist in partition placement.
%%%%%%%%%%%%%%%%%%%%%%%%%%%%
\subsection{Scipy optimize}
  The scipy optimize library provides many tools for optimizing an input function with local and global minimization techniques. Our usage of the optimize library relies on the minimize function, using the basinhopping method as the global optimizer, and the constrained Nelder-Mead method as the local optimizer.
  
\subsubsection{Basinhopping\cite{basinhoppingwales}}
  
\subsubsection{Constrained Nelder-Mead}
  
\subsection{human optimization}
  
  The ``black box'' method using scipy optimize's basinhopping and constrained Nelder-Mead minimizers is functional for small parameter spaces. However, even with modestly large parameter spaces such as the one seen in the Level 2 experiment (Fig. \ref{level2_nocut}), the time-to-solution estimator function is not smooth enough for scipy optimize to honor the constraints or bounds of the problem, leading to the time-to-solution estimator crashing. This lead to the development of an alternative method.  

  The ``human optimization'' method utilizes the geometrical information of the problem to attempt to find optimal cuts. This method prioritizes finding cut line locations that cut along a ``natural boundary''. A natural boundary is a subset boundary that coincides with the geometrical features of the mesh. In Fig. \ref{partitioning_example}, we notice natural boundaries every centimeter in each dimension. 
 \begin{figure}[h]
\centering
\includegraphics[scale=0.2]{../figures/spiderweb_10x10_sparse.png}
\caption{An unstructured mesh with natural boundaries at 1 cm intervals in both dimensions.}
\label{partitioning_example}
\end{figure}

This optimization method:
\begin{enumerate}
  \item Find the most suitable natural boundaries in the $x$ dimension.
  \item For each set of columns, find the most suitable natural boundaries in the $y$ dimension.
  \item Run all iterations of cut lines selected. 
\end{enumerate}

\subsubsection{Finding natural boundaries}

In order to identify natural boundaries, we analyze the detailed cumulative distribution function (CDF) of the vertices in each dimension. The jumps in the CDF correspond to natural boundaries. Figure \ref{vert_cdf} shows the x-vertex CDF of the mesh in Fig. \ref{partitioning_example}. 
\begin{figure}[h]
\centering
\includegraphics[scale=0.75]{../figures/xvertexcdf.pdf}
\caption{The x-vertex CDF of the mesh shown in Fig. \ref{partitioning_example}}.
\label{vert_cdf}
\end{figure}

To identify where the jumps in the CDF occur, we take the gradient of the CDF and isolate the largest discontinuities in it. Figure \ref{gradcdf} plots the gradient of the CDF shown in Fig. \ref{vert_cdf}.
\begin{figure}[h]
\centering
\includegraphics[scale=0.75]{../figures/gradcdf.pdf}
\caption{The gradient of the CDF shown in Fig. \ref{vert_cdf}}.
\label{gradcdf}
\end{figure}
The largest discontinuities in Fig. \ref{gradcdf} occur at the instances where there are natural boundaries all the way through the mesh, or at 1 cm intervals. 


\FloatBarrier
\subsubsection{Finding natural boundaries for sets of columns}
In order to globalize the optimization of the cut lines per column, we set up a binary tree of test cases, such as the one shown in Fig. \ref{binary_tree}. Each layer in the tree represents one set of $y$ cut lines. The first layer, or the root of the tree, represents the case where we try and find the natural boundaries in $y$ throughout all columns, in this case 4 columns. The next layer tries to find two sets of natural boundaries, one set of natural $y$ boundaries through the first two columns, and another set through the final 2 columns. Finally, the last case finds a set of natural $y$ boundaries in each individual column. 

\begin{figure}[h]
\centering
\includegraphics[scale=0.75]{../figures/binary_tree.pdf}
\label{binary_tree}
\caption{A binary tree where each node represents the number of columns we are trying to find a natural boundary through.}
\end{figure}

Let's consider the mesh of the Level 2 experiment shown in Fig. \ref{level2_nocut}. 

\begin{figure}[h]
\centering
\includegraphics[scale=0.3]{../../figures/level2_nocut.png}
\label{level2_nocut}
\caption{The mesh for the Level 2 experiment.}
\end{figure}

\begin{figure}[h]
\centering
\includegraphics[scale=0.55]{../../figures/lvl2_suite_0.pdf}
\includegraphics[scale=0.55]{../../figures/lvl2_suite_1.pdf}
\includegraphics[scale=0.55]{../../figures/lvl2_suite_2.pdf}
\end{figure}

\tcr{Use level 2 mesh to show how the cut lines differ}


\FloatBarrier
\section{Analytical Mesh Results}
Analytical mesh density results will go here.

\section{Centroid-Mesh Results}

PDT mesh results will go here.

\section{PDT Comparison Results}

The PDT comparison goes here.
