\pdfminorversion=4
\documentclass[xcolor={usenames,dvipsnames,svgnames,table}]{beamer}
%for printing or having a crappy pdf reader backup
%\documentclass[handout]{beamer}
\mode<presentation>
\usetheme{Madrid}

\input{common/packages.tex}
\input{common/macros.tex}
%Prevent authblk from loading
\dontusepackage{authblk}


\usecolortheme[RGB={128,128,128}]{structure}
\definecolor{my_gray}{RGB}{105,105,105}
\setbeamercolor{khumna}{fg=white,bg=my_gray}
%teal \usecolortheme[RGB={0,128,128}]{structure}
%\useoutertheme{miniframes}
\useinnertheme{default}
\definecolor{Maroon}{RGB}{80,0,0}
\definecolor{BurntOrange}{RGB}{204,85,0}

\usepackage[figurename=,tablename=]{caption}
\setbeamercolor{normal text}{fg=black}
\setbeamercovered{dynamic}
\beamertemplatetransparentcovereddynamicmedium
\setbeamertemplate{caption}[numbered]
\usepackage{colortbl}
\newcommand {\mathsym}[1]{{}}
\newcommand {\unicode}{{}}
\newcommand{\om}{\boldsymbol{\Omega}}
\newcommand{\etal}{{\it et al.\,}}
\newcommand{\vr}{\vec{r}}
\newcommand{\vo}{\vec{\Omega}}
\newcolumntype{L}{>{\centering\arraybackslash}m{3cm}}
\newcommand{\tcr}[1]{\textcolor{red}{#1}}
%Creating a norm command
\newcommand{\norm}[1]{\left\lVert#1\right\rVert}
%Allow page breaks within align
\allowdisplaybreaks



\newlength \figwidth
\setlength \figwidth {0.5\textwidth}

\newcommand{\backupbegin}{
   \newcounter{finalframe}
   \setcounter{finalframe}{\value{framenumber}}
}
\newcommand{\backupend}{
   \setcounter{framenumber}{\value{finalframe}}
}

\addtobeamertemplate{frametitle}{}
{
	\begin{textblock*}{100mm}(0.85\textwidth,-1cm)
	\includegraphics[height = 1cm,width=2cm]{figures/cert_logo_maroon.png}
	\end{textblock*}
}
\beamertemplatenavigationsymbolsempty



\begin{document}
%	

\title[Partitioning Optimization]{Partitioning Optimization for Massively Parallel Transport Sweeps on Unstructured Grids}
\author[Ghaddar]{Tarek Ghaddar}
\institute[TAMU]{Tarek Ghaddar \\ Chair: Jean Ragusa \\ Committee: Jim Morel, Marvin Adams, Nancy Amato}
\date[December 12, 2018]

{
\setbeamertemplate{headline}{}
\begin{frame}
\vspace{-1.1cm}
	\begin{figure}[t]
		\centering
			\includegraphics[width=.25\textwidth]{figures/cert_logo_maroon.png}
	\end{figure}
\vspace{-0.75cm}
\titlepage
\end{frame}
}

\setbeamertemplate{footline}
{

%\vspace{-0.1ex}
\begin{beamercolorbox}[wd=\textwidth,ht=3.5ex]{khumna}
\includegraphics[width = 0.95\textwidth]{figures/cert_banner.pdf}
	%\begin{textblock*}{10mm}(0.95\textwidth,10cm)
	\insertframenumber/\inserttotalframenumber
	%\end{textblock*}
\end{beamercolorbox}
%\begin{beamercolorbox}[wd=0.05\textwidth,ht=3.5ex]{khumna}

%\end{beamercolorbox}
}

\begin{frame}
\tableofcontents
\end{frame}

\section{Introduction and Previous Work}

\begin{frame}[t]\frametitle{Introduction}
\begin{block}{}
\begin{itemize}
	\item Massively parallel transport sweeps have been shown to scale up to 750,000 cores on logically cartesian grids.
	\item Structured meshes are somewhat limiting when attempting to simulate more complex problems and experiments.
	\item Unstructured meshes allow us to simulate realistic problems, but introduce unbalanced partitions. 
	\item PDT (Texas A\&M's massively parallel transport code) introduced two load balancing algorithms that repartition the mesh in order to obtain a roughly equivalent amount of cells per processor. 
	\item However, this can sacrifice the optimal sweep partitioning (cut lines all the way through the domain) in favor of balance. 
	\item The work proposed will balance perfect load balancing and optimal sweep partitioning in order to achieve the best possible time to solution.
\end{itemize}
\end{block}
\end{frame}

\begin{frame}[t]\frametitle{Load Balancing}
\begin{block}{Load Balance Metric}
  \begin{itemize}
    \item Max cells per subdomain divided by the average cells per subdomain:
      \begin{itemize}
        \item$f =\frac{\underset{ij}{\text{max}}(N_{ij})}{\frac{N_{tot}}{I\cdot J}}$
      \end{itemize}
    \item Column-wise metric: $f_I = \underset{i}{\text{max}}[\sum_{j} N_{ij}]/\frac{N_{tot}}{I}$
    \item Row-wise metric: $f_J = \underset{j}{\text{max}}[\sum_{i} N_{ij}]/\frac{N_{tot}}{J}$
    \item Per Column row-wise metric: $f_{J_i} = \text{max}[N_{ij}]_i/\frac{\sum_{j}N_{ij}}{J_i}$
  \end{itemize}		
  \tcr{Goal: minimize $f$ using locations of cut lines in X and Y}\\
    
  Subsequent improvement in algorithm: once dimension 1 has been balanced, balance dimension 2, then balance dimension 3 (\tcr{load balancing by dimension})
\end{block}
\end{frame}

\begin{frame}[t]\frametitle{No Load Balancing, f = 41.82}
\centering
\includegraphics[scale=0.25]{figures/im12d_nolb.png}
\end{frame}

\begin{frame}[t]\frametitle{Load Balancing, f = 2.97}
\centering
\includegraphics[trim={1cm 0cm 0cm 3cm},clip,scale=0.25]{figures/im12d_oldlb.png}
\end{frame}

\begin{frame}[t]\frametitle{Theoretical Motivation for LBD \\ (Courtesy of Dr. Adams)}
  \begin{block}{}
  \begin{itemize}
    \item Consider simple 2D layout with $M$ unaligned patches of high mesh density
    \item The cellset layout is $[M(N+1)+1] \times [M(N+1)+1]$ but only $MN^2$ cellsets have much work.
    \item Load Imbalance Factor $= \frac{\left( M(N+1)+1 \right)^2}{MN^2} \xrightarrow{N\to \infty} \frac{M^2N^2}{MN^2} = M$
  \end{itemize}
  \end{block}
  \begin{center}
    \includegraphics[width=5cm ]{figures/2dgeneral.png}
  \end{center}
\end{frame}

\begin{frame}[t]\frametitle{Load Balancing By Dimension, f = 2.02}
\centering
\includegraphics[scale=0.22]{figures/im12d_newlb.png}
\end{frame}

\begin{frame}[t]\frametitle{3D Load Balancing By Dimension}
\centering
\includegraphics[trim={0cm 1cm 0cm 3cm},clip,scale=0.27]{figures/im1_foam_448.png}
\end{frame}

\begin{frame}[t]\frametitle{Consequences of Load Balancing By Dimension}
  \begin{block}{}
  \begin{itemize}
    \item Perfect load balance in some cases will come at the cost of optimal sweeping.
    \item Time to solution is the most important parameter, and if keeping a more optimal sweeping grid means a less balanced problem, then so be it.
    \item The concept of a stage may be misleading when dealing with imbalanced partitions, as we cannot easily characterize the idle time.
    \item A time-to-solution estimator must be built to more accurately predict sweep time.
  \end{itemize}
  \end{block}
\end{frame}

\begin{frame}[t]\frametitle{Sweep on Regular Grid with 3 Angle Sets}
	\animategraphics[loop,controls,width=0.8\linewidth]{10}{figures/sweep_figs/sweeps_png/sweep_regular_20x20_as3_dog/sweep_regular_20x20_as3_dog_}{1}{48}
	%\href{run:figures/sweep_figs/sweeps_png/sweep_regular_20x20_as3_dog/animation.gif}{Animation.gif}
\end{frame}

\begin{frame}[t]\frametitle{Sweep on LBD Grid with 3 Angle Sets}
	\animategraphics[loop,controls,width=0.8\linewidth]{10}{figures/sweep_figs/sweeps_png/sweep_random_20x20_as3_dog/sweep_random_20x20_as3_dog_}{1}{101}
	%\href{run:figures/sweep_figs/sweeps_png/sweep_random_20x20_as3_dog/animation.gif}{Animation.gif}
\end{frame}


\begin{frame}[t]\frametitle{Sweep on Worst Grid with 1 Angle Set}
	\animategraphics[loop,controls,width=0.8\linewidth]{10}{figures/sweep_figs/sweeps_png/sweep_worst_20x20_as1_dog/sweep_worst_20x20_as1_dog_}{1}{230}
	%\href{run:figures/sweep_figs/sweeps_png/sweep_worst_20x20_as1_dog/animation.gif}{Animation.gif}
\end{frame}

\section{Partitioning Optimization on Unstructured Grids}

\begin{frame}[t]\frametitle{Overview}
\begin{block}{}
\begin{itemize}
	\item We need to optimize the cut line location not for balance, but for the best possible sweep time.
	\item We must build a time-to-solution estimator that calculates the time to solution for a given cut line partitioning.
	\item The time to solution estimator will be fed into an optimizing function that minimizes the time to solution. The cut lines corresponding to the minimum time to solution are the optimal partitioning scheme.
\end{itemize}
\end{block}
\end{frame}

\begin{frame}[t]\frametitle{Time to Solution Estimator}
\begin{block}{}
\begin{itemize}
  \item For a given partition, this tool will build subset-level (not cell level) directed task dependence graphs for each octant, and weight the edges of that graph based on a set of criteria.
  \item The estimator returns the maximum time to sweep across one of the graphs.
  \item The maximum sweep time corresponds to the graph that has the maximum longest weighted path. 
\end{itemize}
\end{block}
\begin{block}{}
\begin{equation}
\text{TOS} = f(P_x, P_y, P_z, \text{\textcolor{red}{cut lines, threads}, machine params})
\end{equation}
\end{block}
\begin{block}{Important Assumption}
This process assumes that there are NO cycles in the TDGs. All graphs are acyclic.
\end{block}
\end{frame}

\begin{frame}[t]\frametitle{Time to Solution Estimator}
\begin{block}{}
\begin{enumerate}
	\item Given a partitioning scheme (cut lines), build adjacency matrix.
	\item Build Directed Acyclic Graphs (DAGs) from the adjacency matrix, one for each octant.
	\item Weight the DAG's edges based on:
	\begin{itemize}
		\item Solve and communication time of each subset to its neighbors.
		\item Conflicts that arise between DAGs due to the sweep.
		\item This final edge weight reflects the amount of time it takes to solve the base node.
	\end{itemize}
	\item Compute solve time by getting the maximum edge-weighted sum of the longest paths for all DAGs.
\end{enumerate}
\end{block}
\end{frame}

\begin{frame}[t]\frametitle{Weighting the Graphs}
\begin{block}{}
\begin{itemize}
	\item In order to weight the graph properly, we need to have a mesh density function that we can calculate the cells (and then unknowns) per subset from.
	\item The weight of the edge between two nodes (subsets) in the graph represents the solve and communication time of the base node.
\end{itemize}
\end{block}
\begin{block}{}
\begin{align}
\text{cells per subset} &= \int_{\tcr{x_i}}^{\tcr{x_{i+1}}} \int_{\tcr{y_j}}^{\tcr{y_{j+1}}} \int_{\tcr{z_k}}^{\tcr{z_{k+1}}} \text{mesh density } dx dy dz \\
\label{weight}
\text{weight} &= N_u\cdot T_u (\text{\textcolor{red}{threads}}) + N_b\cdot T_{\text{comm}} + \text{latency}\cdot M_L \\
N_u &= \text{num cells}\cdot \text{unknowns per cell} \\
N_b &\approx(\text{num cells})^{\frac{2}{3}}\cdot \text{unknowns per boundary cell}
\end{align}
\end{block}
\end{frame}



\begin{frame}[t]\frametitle{Determining Cells per Subset}
	\centering
	\includegraphics[scale=0.65]{figures/mesh_points.pdf}
\end{frame}

\begin{frame}[t]\frametitle{Determining Cells per Subset}
	\centering
	\includegraphics[scale=0.65]{figures/mesh_subsets.pdf}
\end{frame}

\begin{frame}[t]\frametitle{Determining Cells per Subset}
	\centering
	\includegraphics[scale=0.65]{figures/full_mesh_density.pdf}
\end{frame}

\begin{frame}[t]\frametitle{Universal Edge Weighting}
\centering
\includegraphics[scale=0.5]{figures/printgraph.pdf}
\end{frame}

\begin{frame}[t]\frametitle{Conflict Detection and Resolution}
\begin{block}{}
\begin{itemize}
	\item Once all edges in all TDGs are appropriately weighted according to Eq. \ref{weight}, we need to detect and resolve sweep conflicts between the 8 octants.
	\item For perfectly balanced and optimally partitioned problems (structured grids) we will default to a depth-of-graph conflict resolution.
	\item For unbalanced partitions, we will default to a first come first serve conflict resolution.
    \item We check for conflicts only amongst the longest paths of each graph. This simplification takes into account the upstream dependencies of each node, without losing the accuracy of the time to sweep each graph.
\end{itemize}
\end{block}
\end{frame}

\begin{frame}[t]\frametitle{First Come First Serve Conflict Resolution}
\begin{block}{}
\begin{itemize}
	\item The first octant to arrive to a node will begin solving it, and the remaining octants will incur a delay (if applicable).
	\item The delay is reflected in each remaining TDG by adding the corresponding delay as a weight to the applicable edge.
	\item If two octants arrive to a node at the same time, the octant with the greater remaining depth-of-graph and priority octant wins the tie.
\end{itemize}
\end{block}
\begin{block}{Potential Future Work: hybrid depth-of-graph/first come first serve conflict resolution}
\begin{itemize}
	\item Two (or more) paths arrive to a node at different time, and when that node is free to solve, the path with the greater depth-of-graph remaining solves first.
\end{itemize}
\end{block}
\end{frame}

\end{document}
