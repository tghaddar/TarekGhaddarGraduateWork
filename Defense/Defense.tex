\documentclass[xcolor={usenames,dvipsnames,svgnames,table}]{beamer}

\mode<presentation>
\usetheme{Madrid}

\usecolortheme[RGB={80,0,0}]{structure}
\useoutertheme[subsection=false]{miniframes}
\useinnertheme{default}

% hide navigation controlls
\setbeamertemplate{navigation symbols}{}

\setbeamercolor{normal text}{fg=black}
\setbeamercovered{dynamic}
\beamertemplatetransparentcovereddynamicmedium
%\usepackage{chronology}
\setbeamertemplate{caption}[numbered]

\definecolor{Maroon}{RGB}{80,0,0}
\definecolor{BurntOrange}{RGB}{204,85,0}

% load macros and prevent authblk from loading
\input{common/macros.tex}
\dontusepackage{authblk}

% load packages, settings and definitions
\input{common/packages.tex}
\input{common/settings.tex}
\input{common/definitions.tex}

% nicer item settings
\setlist[1]{nolistsep,label=\(\textcolor{Maroon}{\blacksquare}\)}
\setlist[2]{nolistsep,label=\(\textcolor{Maroon}{\bullet}\)}

\newcommand {\mathsym}[1]{{}}
\newcommand {\unicode}{{}}
\newcommand{\om}{\boldsymbol{\Omega}}
\newcommand{\etal}{{\it et al.\,}}
\newcommand{\vr}{\vec{r}}
\newcommand{\vo}{\vec{\Omega}}
\newcolumntype{L}{>{\centering\arraybackslash}m{3cm}}
\newcommand{\tcr}[1]{\textcolor{red}{#1}}

\setenumerate[1]{
	label=\protect\usebeamerfont{enumerate item}%
	\protect\usebeamercolor[fg]{enumerate item}%
	\insertenumlabel.
}

%%%%%%%%%%%%%%%%%%%%%%%%%%%%%%%%%%%%%%%%%%%%%%%
%%% edit to fit your document

% set up pdf support and indexing
\hypersetup{
    pdftitle={<Title>},
    pdfauthor={<author>},
    pdfsubject={<subject>},
    pdfkeywords={<keywords>},
}

\title[Partitioning Optimization]{Partitioning Optimization for Massively Parallel Transport Sweeps on Unstructured Grids}
\author[Ghaddar]{Tarek Habib Ghaddar \\ \small{Chair: Jean C. Ragusa \\ Committee: Marvin L. Adams, Nancy Amato, Jim E. Morel}}
\institute[Texas A\&M]{Department of Nuclear Engineering \\ Texas A\&M University}
\date[June 19, 2019]

\begin{document}

% title page, do not edit
{
\setbeamertemplate{headline}[default] 
\begin{frame}
\vspace{-1.1cm}
	\begin{figure}[t]
		\centering
			\includegraphics[width=.25\textwidth]{images/seal.png}
	\end{figure}
\vspace{-0.75cm}
\titlepage
\end{frame}
}

\begin{frame}
\tableofcontents
\end{frame}

%%%%%%%%%%%%%%%%%%%%%%%%%%%%%%%%%%%%%%%%%%%%%%%%%%%%%%%%%%%%%%%%%%
%%%%%%%%%%%%%%%%%%%%%%%%%%%%%%%%%%%%%%%%%%%%%%%%%%%%%%%%%%%%%%%%%%
%%%%%%%%%%%%%%%%%%%%%%%%%%%%%%%%%%%%%%%%%%%%%%%%%%%%%%%%%%%%%%%%%%
\section{Introduction and Background}
\subsection{}
%%%%%%%%%%%%%%%%%%%%%%%%%%%%%%%%%%%%%%%%%%%%%%%%%%%%%%%%%%%%%%%%%%
%%%%%%%%%%%%%%%%%%%%%%%%%%%%%%%%%%%%%%%%%%%%%%%%%%%%%%%%%%%%%%%%%%
%%%%%%%%%%%%%%%%%%%%%%%%%%%%%%%%%%%%%%%%%%%%%%%%%%%%%%%%%%%%%%%%%%

\begin{frame}[t]\frametitle{Introduction}
\begin{block}{}
\begin{itemize}
	\item Massively parallel transport sweeps have been shown to scale up to 750,000 cores on logically cartesian grids.
	\item Structured meshes are somewhat limiting when attempting to simulate more complex problems and experiments.
	\item Unstructured meshes allow us to simulate realistic problems, but introduce unbalanced partitions. 
	\item PDT (Texas A\&M's massively parallel transport code) introduced two load balancing algorithms that repartition the mesh in order to obtain a roughly equivalent amount of cells per processor. 
	\item However, this can sacrifice the optimal sweep partitioning (cut lines all the way through the domain) in favor of balance. 
	\item The method described will balance perfect load balancing and optimal sweep partitioning in order to achieve the best possible time to solution.
\end{itemize}
\end{block}
\end{frame}

\begin{frame}[t]\frametitle{Transport Sweeps}
\begin{block}{}
\begin{equation}
\vo_m \cdot \vec\nabla \psi_m^{(l+1)}(\vr) + \Sigma_t \psi_m^{(l+1)}(\vr) = q_m^{(l)}(\vr),
\label{iteration}
\end{equation}
\begin{itemize}
\item The domain is meshed and discretized, allowing one cell at a time to be solved.
\item The solution across a cell interface is connected based on an upwind approach.
\end{itemize}
\end{block}
\centering
\includegraphics[scale = 0.15]{../figures/UnstructureMesh.pdf}
\includegraphics[scale = 0.15]{../figures/StructuredMesh.pdf}
\end{frame}

\begin{frame}[t]\frametitle{Parallel Transport Sweeps}

\begin{block}{A parallel sweep algorithm is defined by three properties:}
\begin{itemize}
\item partitioning: dividing the domain among available processors,
\item aggregation: grouping cells, directions, and energy groups into tasks,
\item scheduling: choosing which task to execute if more than one is available.
\end{itemize}
\end{block}
\end{frame}


\begin{frame}[t]\frametitle{KBA Algorithm}
\begin{block}{}
\begin{itemize}
\item KBA limits the number of processors in $z$ to one, solves one group at a time, and does not aggregate in angle or group.
\item KBA primarily uses a "successive in angle, successive in quadrant" approach.
\item An octant pipelines its angular work, and once all directions are complete, the opposing octant pipelines them back.
\item This is then done for all remaining octant pairs.
\end{itemize}
\end{block}
\end{frame}

\begin{frame}[t]\frametitle{KBA Algorithm}
\begin{block}{}
\begin{itemize}
\item KBA utlilizes a "pipelining" or assembly line approach where new work is started before old work is fully completed.
\end{itemize}
\end{block}
\centering
\includegraphics[scale=0.6,trim={0cm 1cm 0cm 0cm},clip]{../figures/pipeline_example.pdf}
\end{frame}

\begin{frame}[t]\frametitle{Sweeps in PDT}
\begin{block}{}
\begin{itemize}
\item PDT's extension of the KBA algorithm does not limit $P_z, A_m, G,$ or $A_g$.
\item PDT also launches all 8 octants (4 quadrants in 2D) simultaneously, rather than one octant-pair at a time.
\item Unlike KBA, PDT must schedule around conflicts that emerge between octants.
\end{itemize}
\end{block}
\end{frame}

\begin{frame}[t]\frametitle{Tie Breaking in PDT}

\begin{block}{If two or more tasks reach a processor at the same time, PDT employs a tie breaking strategy:}

\begin{enumerate}
	\item The task with the greater depth-of-graph remaining (simply, more work remaining) goes first.
	\item If the depth-of-graph remaining is tied, the task with $\Omega_x > 0$ wins.
	\item If multiple tasks have $\Omega_x > 0$, then the task with $\Omega_y > 0$ wins.
	\item If multiple tasks have $\Omega_y > 0$, then the task with $\Omega_z > 0$ wins.
\end{enumerate}
\end{block}
\end{frame}

\begin{frame}[t]\frametitle{Unstructured Meshing in PDT}
  \begin{block}{}
    \begin{itemize}
      \item PDT using unstructured meshes has been a priority since early 2014. 
      \item Unstructured meshes allow for simulation of a wider and more general variety of problems.
      \item Three unstructured mesh types are supported in PDT:
        \begin{itemize}
          \item Triangle (2D and 2D extruded triangular/prismatic meshes).
          \item Spiderweb (2D and 2D extruded prismatic meshes).
          \item Cubit/OpenFOAM (fully unstructured 3D meshes).
        \end{itemize}
    \end{itemize}
  \end{block}
\end{frame}

\begin{frame}[t]\frametitle{Unstructured Meshing in PDT}
\begin{block}{Partitioning for Unstructured Meshes}
\begin{itemize}
\item ``Cut lines" in 2D (cut planes for 3D) are used to slice through the mesh in the $x$, $y$, and $z$ dimensions.
\item The cut planes form brick partitions, called subsets, that have unstructured meshes inside of them. 
\item The subsets are distributed amongst the processor domain.
\end{itemize}
\end{block}
\end{frame}

\begin{frame}[t]\frametitle{Partitioning Example}
\centering
\includegraphics[scale=0.45,trim={0.95in 0.64in 0.35in 0.44in},clip]{../figures/partitioning_example.pdf}
\end{frame}

\end{document}