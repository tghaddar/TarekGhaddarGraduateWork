\documentclass[xcolor={usenames,dvipsnames,svgnames,table}]{beamer}

\mode<presentation>
\usetheme{Madrid}

\usecolortheme[RGB={80,0,0}]{structure}
\useoutertheme[subsection=false]{miniframes}
\useinnertheme{default}

% hide navigation controlls
\setbeamertemplate{navigation symbols}{}

\setbeamercolor{normal text}{fg=black}
\setbeamercovered{dynamic}
\beamertemplatetransparentcovereddynamicmedium
%\usepackage{chronology}
\setbeamertemplate{caption}[numbered]

\definecolor{Maroon}{RGB}{80,0,0}
\definecolor{BurntOrange}{RGB}{204,85,0}

% load macros and prevent authblk from loading
\input{common/macros.tex}
\dontusepackage{authblk}

% load packages, settings and definitions
\input{common/packages.tex}
\input{common/settings.tex}
\input{common/definitions.tex}

% nicer item settings
\setlist[1]{nolistsep,label=\(\textcolor{Maroon}{\blacksquare}\)}
\setlist[2]{nolistsep,label=\(\textcolor{Maroon}{\bullet}\)}

\newcommand {\mathsym}[1]{{}}
\newcommand {\unicode}{{}}
\newcommand{\om}{\boldsymbol{\Omega}}
\newcommand{\etal}{{\it et al.\,}}
\newcommand{\vr}{\vec{r}}
\newcommand{\vo}{\vec{\Omega}}
\newcolumntype{L}{>{\centering\arraybackslash}m{3cm}}
\newcommand{\tcr}[1]{\textcolor{red}{#1}}

\setenumerate[1]{
	label=\protect\usebeamerfont{enumerate item}%
	\protect\usebeamercolor[fg]{enumerate item}%
	\insertenumlabel.
}

%%%%%%%%%%%%%%%%%%%%%%%%%%%%%%%%%%%%%%%%%%%%%%%
%%% edit to fit your document

% set up pdf support and indexing
\hypersetup{
    pdftitle={<Title>},
    pdfauthor={<author>},
    pdfsubject={<subject>},
    pdfkeywords={<keywords>},
}

\title[Partitioning Optimization]{Partitioning Optimization for Massively Parallel Transport Sweeps on Unstructured Grids}
\author[Ghaddar]{Tarek Habib Ghaddar \\ \small{Chair: Jean C. Ragusa \\ Committee: Marvin L. Adams, Nancy Amato, Jim E. Morel}}
\institute[Texas A\&M]{Department of Nuclear Engineering \\ Texas A\&M University}
\date[June 19, 2019]

\begin{document}

% title page, do not edit
{
\setbeamertemplate{headline}[default] 
\begin{frame}
\vspace{-1.1cm}
	\begin{figure}[t]
		\centering
			\includegraphics[width=.25\textwidth]{images/seal.png}
	\end{figure}
\vspace{-0.75cm}
\titlepage
\end{frame}
}

\begin{frame}
\tableofcontents
\end{frame}

%%%%%%%%%%%%%%%%%%%%%%%%%%%%%%%%%%%%%%%%%%%%%%%%%%%%%%%%%%%%%%%%%%
%%%%%%%%%%%%%%%%%%%%%%%%%%%%%%%%%%%%%%%%%%%%%%%%%%%%%%%%%%%%%%%%%%
%%%%%%%%%%%%%%%%%%%%%%%%%%%%%%%%%%%%%%%%%%%%%%%%%%%%%%%%%%%%%%%%%%
\section{Introduction and Background}
\subsection{}
%%%%%%%%%%%%%%%%%%%%%%%%%%%%%%%%%%%%%%%%%%%%%%%%%%%%%%%%%%%%%%%%%%
%%%%%%%%%%%%%%%%%%%%%%%%%%%%%%%%%%%%%%%%%%%%%%%%%%%%%%%%%%%%%%%%%%
%%%%%%%%%%%%%%%%%%%%%%%%%%%%%%%%%%%%%%%%%%%%%%%%%%%%%%%%%%%%%%%%%%

\begin{frame}[t]\frametitle{Introduction}
\begin{block}{}
\begin{itemize}
	\item Massively parallel transport sweeps have been shown to scale up to 750,000 cores on logically cartesian grids.
	\item Structured meshes are somewhat limiting when attempting to simulate more complex problems and experiments.
	\item Unstructured meshes allow us to simulate realistic problems, but introduce unbalanced partitions. 
	\item PDT (Texas A\&M's massively parallel transport code) introduced two load balancing algorithms that repartition the mesh in order to obtain a roughly equivalent amount of cells per processor. 
	\item However, this can sacrifice the optimal sweep partitioning (cut lines all the way through the domain) in favor of balance. 
	\item The method described will balance perfect load balancing and optimal sweep partitioning in order to achieve the best possible time to solution.
\end{itemize}
\end{block}
\end{frame}

\begin{frame}[t]\frametitle{Transport Sweeps}
\begin{block}{}
\begin{equation}
\vo_m \cdot \vec\nabla \psi_m^{(l+1)}(\vr) + \Sigma_t \psi_m^{(l+1)}(\vr) = q_m^{(l)}(\vr),
\label{iteration}
\end{equation}
\begin{itemize}
\item The domain is meshed and discretized, allowing one cell at a time to be solved.
\item The solution across a cell interface is connected based on an upwind approach.
\end{itemize}
\end{block}
\centering
\includegraphics[scale = 0.15]{../figures/UnstructureMesh.pdf}
\includegraphics[scale = 0.15]{../figures/StructuredMesh.pdf}
\end{frame}

\begin{frame}[t]\frametitle{Parallel Transport Sweeps}

\begin{block}{A parallel sweep algorithm is defined by three properties:}
\begin{itemize}
\item partitioning: dividing the domain among available processors,
\item aggregation: grouping cells, directions, and energy groups into tasks,
\item scheduling: choosing which task to execute if more than one is available.
\end{itemize}
\end{block}
\end{frame}


\begin{frame}[t]\frametitle{KBA Algorithm}
\begin{block}{}
\begin{itemize}
\item KBA limits the number of processors in $z$ to one, solves one group at a time, and does not aggregate in angle or group.
\item KBA primarily uses a "successive in angle, successive in quadrant" approach.
\item An octant pipelines its angular work, and once all directions are complete, the opposing octant pipelines them back.
\item This is then done for all remaining octant pairs.
\end{itemize}
\end{block}
\end{frame}

\begin{frame}[t]\frametitle{KBA Algorithm}
\begin{block}{}
\begin{itemize}
\item KBA utlilizes a "pipelining" or assembly line approach where new work is started before old work is fully completed.
\end{itemize}
\end{block}
\centering
\includegraphics[scale=0.6,trim={0cm 1cm 0cm 0cm},clip]{../figures/pipeline_example.pdf}
\end{frame}

\begin{frame}[t]\frametitle{Sweeps in PDT}
\begin{block}{}
\begin{itemize}
\item PDT's extension of the KBA algorithm does not limit $P_z, A_m, G,$ or $A_g$.
\item PDT also launches all 8 octants (4 quadrants in 2D) simultaneously, rather than one octant-pair at a time.
\item Unlike KBA, PDT must schedule around conflicts that emerge between octants.
\end{itemize}
\end{block}
\end{frame}

\begin{frame}[t]\frametitle{Tie Breaking in PDT}

\begin{block}{If two or more tasks reach a processor at the same time, PDT employs a tie breaking strategy:}

\begin{enumerate}
	\item The task with the greater depth-of-graph remaining (simply, more work remaining) goes first.
	\item If the depth-of-graph remaining is tied, the task with $\Omega_x > 0$ wins.
	\item If multiple tasks have $\Omega_x > 0$, then the task with $\Omega_y > 0$ wins.
	\item If multiple tasks have $\Omega_y > 0$, then the task with $\Omega_z > 0$ wins.
\end{enumerate}
\end{block}
\end{frame}

\begin{frame}[t]\frametitle{Unstructured Meshing in PDT}
  \begin{block}{}
    \begin{itemize}
      \item PDT using unstructured meshes has been a priority since early 2014. 
      \item Unstructured meshes allow for simulation of a wider and more general variety of problems.
      \item Three unstructured mesh types are supported in PDT:
        \begin{itemize}
          \item Triangle (2D and 2D extruded triangular/prismatic meshes).
          \item Spiderweb (2D and 2D extruded prismatic meshes).
          \item Cubit/OpenFOAM (fully unstructured 3D meshes).
        \end{itemize}
    \end{itemize}
  \end{block}
\end{frame}

\begin{frame}[t]\frametitle{Unstructured Meshing in PDT}
\begin{block}{Partitioning for Unstructured Meshes}
\begin{itemize}
\item ``Cut lines" in 2D (cut planes for 3D) are used to slice through the mesh in the $x$, $y$, and $z$ dimensions.
\item The cut planes form brick partitions, called subsets, that have unstructured meshes inside of them. 
\item The subsets are distributed amongst the processor domain.
\end{itemize}
\end{block}
\end{frame}

\begin{frame}[t]\frametitle{Partitioning Example}
\centering
\includegraphics[scale=0.45,trim={0.95in 0.64in 0.35in 0.44in},clip]{../figures/partitioning_example.pdf}
\end{frame}


\section{Load Balancing}
\subsection{}

\begin{frame}[t]\frametitle{Load Balancing}
\begin{block}{Load Balance Metric}
  \begin{itemize}
    \item Max cells per subdomain divided by the average cells per subdomain:
      \begin{itemize}
        \item$f =\frac{\underset{ij}{\text{max}}(N_{ij})}{\frac{N_{tot}}{I\cdot J}}$
      \end{itemize}
    \item Column-wise metric: $f_I = \underset{i}{\text{max}}[\sum_{j} N_{ij}]/\frac{N_{tot}}{I}$
    \item Row-wise metric: $f_J = \underset{j}{\text{max}}[\sum_{i} N_{ij}]/\frac{N_{tot}}{J}$
    \item Per Column row-wise metric: $f_{J_i} = \text{max}[N_{ij}]_i/\frac{\sum_{j}N_{ij}}{J_i}$
  \end{itemize}		
  \tcr{Goal: minimize $f$ using locations of cut lines in X and Y}\\
    
  Subsequent improvement in algorithm: once dimension 1 has been balanced, balance dimension 2, then balance dimension 3 (\tcr{load balancing by dimension})
\end{block}
\end{frame}
\begin{frame}[t]\frametitle{Load Balancing Metrics}
\begin{equation}
f =\frac{\underset{ijk}{\text{max}}(N_{ijk})}{\frac{N_{tot}}{I\cdot J\cdot K}},
\label{metric_def}
\end{equation}
\begin{align}
f_{D1} &= \underset{d1}{\text{max}}[\sum_{d2,d3} N_{ijk}]/\frac{N_{tot}}{D1}, \label{f_d1} \\
f_{D2,d1} &= \Big(\underset{d2}{\text{max}}[\sum_{d3} N_{ijk,d1}]/\frac{N_{d1}}{D2}\Big), \label{f_d2}\\
f_{D3,d2,d1} &= \Big( \underset{d3}{\text{max}}[ N_{ijk,d1,d2}]/\frac{N_{d1,d2}}{D3} \Big) . \label{f_d3}
\end{align}
\end{frame}

\begin{frame}[t]\frametitle{Original Load Balancing Algorithm}
\begin{algorithm}[H]
\label{initial_algorithm}
\begin{algorithmic}

\WHILE{$f > \text{tol}_{\text{subset}}$}
  \IF {$f_I > \text{tol}_{\text{col}}$}
    \STATE Redistribute the X cut planes.
  \ENDIF
  \IF {$f_J > \text{tol}_{\text{row}}$}
  	\STATE Redistribute the Y cut planes.
  \ENDIF
\ENDWHILE
\end{algorithmic}
\end{algorithm}
\end{frame}

\begin{frame}[t]\frametitle{Theoretical Motivation for LBD}
  \begin{block}{}
  \begin{itemize}
    \item Consider simple 2D layout with $M$ unaligned patches of high mesh density
    \item The cellset layout is $[M(N+1)+1] \times [M(N+1)+1]$ but only $MN^2$ cellsets have much work.
    \item Load Imbalance Factor $= \frac{\left( M(N+1)+1 \right)^2}{MN^2} \xrightarrow{N\to \infty} \frac{M^2N^2}{MN^2} = M$
  \end{itemize}
  \end{block}
  \begin{center}
    \includegraphics[width=5cm ]{../figures/2dgeneral.png}
  \end{center}
\end{frame}

\begin{frame}[t]\frametitle{Load Balancing By Dimension Algorithm}
\begin{algorithm}[H]
\label{lbd}
\begin{algorithmic}

  \WHILE {$f_{D1} > \text{tol}_{\text{D1}}$}
    \STATE Redistribute the D1 cut planes.
  \ENDWHILE  
  
  \FOR {$d1$ in $D1$}
    \WHILE {$f_{D2,d1} > \text{tol}_{\text{D2}}$}
      \STATE Redistribute the D2 cut planes within d1. 
    \ENDWHILE
  \ENDFOR
  
  \FOR{$d1$ in $D1$}
    \FOR{$d2$ in $D2$}
      \WHILE {$f_{D3,d2,d1} > \text{tol}_{\text{D3}}$ }
        \STATE Redistribute the D3 cut planes within d2 within d1. 
      \ENDWHILE
    \ENDFOR
  \ENDFOR
  
  \STATE Calculate $f$.
\end{algorithmic}
\end{algorithm}
\end{frame}

\begin{frame}[t]\frametitle{Redistribution}
\begin{figure}[H]
\centering
\includegraphics[scale=0.3]{../figures/redistribute_before.pdf}
\includegraphics[scale=0.3]{../figures/redistribute_after.pdf}
\caption{The use of the CDF of triangles per column to redistribute the cut planes in X.}
\label{redistribute}
\end{figure}
\end{frame}


\begin{frame}[t]\frametitle{No Load Balancing, f = 41.82}
\centering
\includegraphics[scale=0.25]{../figures/im12d_nolb.png}
\end{frame}

\begin{frame}[t]\frametitle{Load Balancing, f = 2.97}
\centering
\includegraphics[trim={1cm 0cm 0cm 3cm},clip,scale=0.25]{../figures/im12d_oldlb.png}
\end{frame}


\begin{frame}[t]\frametitle{Load Balancing By Dimension, f = 2.02}
\centering
\includegraphics[scale=0.22]{../figures/im12d_newlb.png}
\end{frame}

\begin{frame}[t]\frametitle{3D Load Balancing By Dimension}
\centering
\includegraphics[trim={0cm 1cm 0cm 3cm},clip,scale=0.23]{../figures/im1_foam_448.png}
\end{frame}

\begin{frame}[t]\frametitle{Load Balancing Study}
\begin{figure}[H]
\centering
\includegraphics[scale=0.25,trim={0.95in 0.64in 0.35in 0.44in},clip]{../figures/og_lb_example.pdf}
\includegraphics[scale=0.25,trim={0.95in 0.64in 0.35in 0.44in},clip]{../figures/lbd_example.pdf}
\label{alg_illustration}
\end{figure}
  \begin{block}{Parametric Study Parameters}
    \begin{itemize}
      \item Number of subsets ranging from 2 to 10 each in $x$ and $y$.
      \item Maximum triangle area ranging from coarsest possible to 0.01 cm\textsuperscript{2}.
      \item Minimum triangle angle kept constant at $20^{\circ}$.
    \end{itemize}
  \end{block}
\end{frame}

\begin{frame}[t]\frametitle{Load Balancing Study}
\begin{table}[H]
\centering
\caption{\bf The percent improvement of the original load balancing algorithm (left) and the load balancing by dimension algorithm (right).}
\scalebox{0.4}{
\begin{tabular}{c|c|c|c|c|c|c|c|c|c} 

\bf Area, $N^{1/2}$ & \bf  2 & \bf 3    &  \bf  4   &  \bf  5   &  \bf 6    &  \bf  7   &   \bf 8   &  \bf 9    &  \bf 10   \\ \hline \hline
\bf Coarse& 0.000 & 0.367 & 0.403 & 0.552 & 0.628 & 0.491 & 0.890 & 0.720 & 0.765 \\ \hline 
 \bf 1.8& 0.000 & 0.091 & 0.337 & 0.364 & 0.473 & 0.390 & 0.767 & 0.413 & 0.683 \\ \hline 
 \bf 1.6& 0.000 & 0.093 & 0.398 & 0.368 & 0.499 & 0.370 & 0.815 & 0.353 & 0.774 \\ \hline 
 \bf 1.4& 0.000 & 0.061 & 0.080 & 0.410 & 0.415 & 0.412 & 0.570 & 0.413 & 0.545 \\ \hline 
 \bf 1.2& 0.000 & 0.007 & 0.391 & 0.340 & 0.378 & 0.315 & 0.536 & 0.245 & 0.196 \\ \hline 
 \bf 1.0& 0.000 & 0.038 & 0.206 & 0.420 & 0.341 & 0.186 & 0.696 & 0.201 & 0.160 \\ \hline 
 \bf 0.8& 0.000 & 0.049 & 0.109 & 0.336 & 0.434 & 0.139 & 0.637 & 0.228 & 0.000 \\ \hline 
 \bf 0.6& 0.000 & 0.000 & 0.057 & 0.199 & 0.163 & 0.346 & 0.517 & 0.000 & 0.090 \\ \hline 
 \bf 0.4& 0.000 & 0.000 & 0.065 & 0.013 & 0.267 & 0.147 & 0.528 & 0.179 & 0.000 \\ \hline 
 \bf 0.2& 0.000 & 0.000 & 0.000 & 0.000 & 0.001 & 0.041 & 0.566 & 0.121 & 0.000 \\ \hline 
 \bf 0.1& 0.000 & 0.000 & 0.000 & 0.000 & 0.000 & 0.000 & 0.540 & 0.089 & 0.000 \\ \hline 
 \bf 0.08&0.000 & 0.000 & 0.000 & 0.000 & 0.000 & 0.000 & 0.458 & 0.000 & 0.000 \\ \hline 
 \bf 0.06&0.000 & 0.000 & 0.000 & 0.000 & 0.000 & 0.000 & 0.409 & 0.000 & 0.000 \\ \hline 
 \bf 0.05&0.000 & 0.000 & 0.000 & 0.000 & 0.000 & 0.000 & 0.360 & 0.000 & 0.000 \\ \hline 
 \bf 0.04&0.000 & 0.000 & 0.000 & 0.000 & 0.000 & 0.000 & 0.348 & 0.000 & 0.000 \\ \hline 
 \bf 0.03&0.000 & 0.000 & 0.000 & 0.000 & 0.000 & 0.000 & 0.293 & 0.000 & 0.000 \\ \hline 
 \bf 0.02&0.000 & 0.000 & 0.000 & 0.000 & 0.000 & 0.000 & 0.000 & 0.000 & 0.000 \\ \hline 
 \bf 0.01&0.000 & 0.000 & 0.000 & 0.000 & 0.000 & 0.000 & 0.000 & 0.000 & 0.000 \\ \hline 
\end{tabular}}
\scalebox{0.4}{
\begin{tabular}{c|c|c|c|c|c|c|c|c|c} 
\bf Area, $N^{1/2}$ & \bf  2 & \bf 3    &  \bf  4   &  \bf  5   &  \bf 6    &  \bf  7   &   \bf 8   &  \bf 9    &  \bf 10   \\ \hline \hline
\bf Coarse& 0.175 & 0.663 & 0.743 & 0.781 & 0.796 & 0.757 & 0.938 & 0.760 & 0.820 \\ \hline 
  \bf 1.8& 0.266 & 0.417 & 0.511 & 0.661 & 0.766 & 0.665 & 0.896 & 0.647 & 0.512 \\ \hline 
  \bf 1.6& 0.262 & 0.426 & 0.568 & 0.635 & 0.760 & 0.631 & 0.897 & 0.542 & 0.557 \\ \hline 
  \bf 1.4& 0.244 & 0.369 & 0.497 & 0.618 & 0.769 & 0.595 & 0.901 & 0.722 & 0.741 \\ \hline 
  \bf 1.2& 0.242 & 0.336 & 0.552 & 0.614 & 0.663 & 0.583 & 0.886 & 0.208 & 0.597 \\ \hline 
  \bf 1.0& 0.203 & 0.287 & 0.458 & 0.549 & 0.442 & 0.605 & 0.875 & 0.536 & 0.552 \\ \hline 
  \bf 0.8& 0.162 & 0.330 & 0.435 & 0.460 & 0.638 & 0.542 & 0.888 & 0.393 & 0.000 \\ \hline 
  \bf 0.6& 0.122 & 0.291 & 0.447 & 0.460 & 0.503 & 0.610 & 0.844 & 0.267 & 0.058 \\ \hline 
  \bf 0.4& 0.093 & 0.274 & 0.310 & 0.400 & 0.488 & 0.519 & 0.888 & 0.328 & 0.000 \\ \hline 
  \bf 0.2& 0.042 & 0.147 & 0.185 & 0.267 & 0.344 & 0.299 & 0.810 & 0.349 & 0.025 \\ \hline 
  \bf 0.1& 0.026 & 0.067 & 0.109 & 0.156 & 0.144 & 0.210 & 0.735 & 0.367 & 0.000 \\ \hline 
  \bf 0.08&0.002 & 0.032 & 0.059 & 0.060 & 0.122 & 0.150 & 0.699 & 0.268 & 0.041 \\ \hline 
  \bf 0.06&0.005 & 0.014 & 0.036 & 0.057 & 0.094 & 0.073 & 0.643 & 0.246 & 0.058 \\ \hline 
  \bf 0.05&0.006 & 0.017 & 0.006 & 0.005 & 0.033 & 0.068 & 0.579 & 0.208 & 0.008 \\ \hline 
  \bf 0.04&0.002 & 0.008 & 0.009 & 0.000 & 0.011 & 0.022 & 0.544 & 0.168 & 0.000 \\ \hline 
  \bf 0.03&0.002 & 0.000 & 0.005 & 0.024 & 0.028 & 0.027 & 0.479 & 0.089 & 0.000 \\ \hline 
  \bf 0.02&0.003 & 0.002 & 0.001 & 0.000 & 0.001 & 0.004 & 0.361 & 0.070 & 0.000 \\ \hline 
  \bf 0.01&0.002 & 0.003 & 0.002 & 0.000 & 0.001 & 0.000 & 0.196 & 0.027 & 0.000 \\ \hline 

\end{tabular}}
\label{all_improvements}
\end{table}
\end{frame}

\begin{frame}[t]\frametitle{Paramtric Study Conclusions}
  \begin{block}{}
  \begin{itemize}
    \item The more uniformly refined your mesh, the more inherently balanced it is.
    \item With the exception of a few outliers, the load balancing by dimension algorithm was an improvement over the original load balancing algorithm.
    \item The metric improved by a max of 76.9\% and a mean of 21.7\% with the load balancing by dimension algorithm over the original load balancing algorithm. 
  \end{itemize}
  \end{block}
\end{frame}

\begin{frame}[t]\frametitle{Consequences of Load Balancing By Dimension}
  \begin{block}{}
  \begin{itemize}
    \item Perfect load balance in some cases will come at the cost of optimal sweeping.
    \item Time to solution is the most important parameter, and if keeping a more optimal sweeping grid means a less balanced problem, then so be it.
    \item The concept of a stage may be misleading when dealing with imbalanced partitions, as we cannot easily characterize the idle time.
    \item A time-to-solution estimator must be built to more accurately predict sweep time.
  \end{itemize}
  \end{block}
\end{frame}

\begin{frame}[t]\frametitle{Sweep on Regular Grid with 3 Angle Sets}
    \centering
	\animategraphics[loop,controls,width=0.7\linewidth]{10}{../figures/sweeps_png/sweep_regular_20x20_as3_dog/sweep_regular_20x20_as3_dog_}{1}{48}
	%\href{run:figures/sweep_figs/sweeps_png/sweep_regular_20x20_as3_dog/animation.gif}{Animation.gif}
\end{frame}

\begin{frame}[t]\frametitle{Sweep on LBD Grid with 3 Angle Sets}
   \centering
	\animategraphics[loop,controls,width=0.7\linewidth]{10}{../figures/sweeps_png/sweep_random_20x20_as3_dog/sweep_random_20x20_as3_dog_}{1}{101}
	%\href{run:figures/sweep_figs/sweeps_png/sweep_random_20x20_as3_dog/animation.gif}{Animation.gif}
\end{frame}


\begin{frame}[t]\frametitle{Sweep on Worst Grid with 1 Angle Set}
    \centering
	\animategraphics[loop,controls,width=0.7\linewidth]{10}{../figures/sweeps_png/sweep_worst_20x20_as1_dog/sweep_worst_20x20_as1_dog_}{1}{230}
	%\href{run:figures/sweep_figs/sweeps_png/sweep_worst_20x20_as1_dog/animation.gif}{Animation.gif}
\end{frame}

%%%%%%%%%%%%%%%%%%%%%%%%%%%%%%%%%%%%%%%%%%%%%%%%%%%%%%%%%%%%%%%%%%%%%%%%%%%%%%%%%%%%%%%
%%%%%%%%%%%%%%%%%%%%%%%%%%%%%%%%%%%%%%%%%%%%%%%%%%%%%%%%%%%%%%%%%%%%%%%%%%%%%%%%%%%%%%%
%%%%%%%%%%%%%%%%%%%%%%%%%%%%%%%%%%%%%%%%%%%%%%%%%%%%%%%%%%%%%%%%%%%%%%%%%%%%%%%%%%%%%%%
\section{Time-To-Solution Estimator}
\subsection{}
%%%%%%%%%%%%%%%%%%%%%%%%%%%%%%%%%%%%%%%%%%%%%%%%%%%%%%%%%%%%%%%%%%%%%%%%%%%%%%%%%%%%%%%
%%%%%%%%%%%%%%%%%%%%%%%%%%%%%%%%%%%%%%%%%%%%%%%%%%%%%%%%%%%%%%%%%%%%%%%%%%%%%%%%%%%%%%%
%%%%%%%%%%%%%%%%%%%%%%%%%%%%%%%%%%%%%%%%%%%%%%%%%%%%%%%%%%%%%%%%%%%%%%%%%%%%%%%%%%%%%%%

\begin{frame}[t]\frametitle{Overview}
\begin{block}{}
\begin{itemize}
	\item We need to optimize the cut line location not for balance, but for the best possible sweep time.
	\item We must build a time-to-solution estimator that calculates the time to solution for a given cut line partitioning.
	\item The time to solution estimator will be fed into an optimizing function that minimizes the time to solution. The cut lines corresponding to the minimum time to solution are the optimal partitioning scheme.
\end{itemize}
\end{block}
\end{frame}

\begin{frame}[t]\frametitle{Time To Solution Estimator}
\begin{block}{}
\begin{enumerate}
    \item Given a partitioning scheme, build an adjacency matrix.
    \item From the adjacency matrix, build Directed Acyclic Graphs (DAGs), one for each quadrant/octant.
    \item Weight each DAG's edges based on the solve and communication time of each subset to its neighbors.
    \item Adjust the weights of each graph to reflect a universal timescale.
    \item Adjust the weights of each graph based on the number of anglesets.
    \item Adjust the weights of each graph to reflect sweep conflicts between octants.
    \item Obtain the time-to-solution.
\end{enumerate}
\end{block}
\end{frame}

\begin{frame}[t]\frametitle{Building the Adjacency Matrix}

\end{frame}


\end{document}